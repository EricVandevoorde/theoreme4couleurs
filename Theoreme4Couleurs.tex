\chapter{Théorème des quatre couleurs}

Nous avons à présent à notre disposition tous les outils nécessaires pour la démonstration complète du théorème des quatre couleurs. Assez ironiquement, la première tentative de démonstration par \textsc{Kempe} en 1879 contient toutes les idées maitresses de la preuve complète.\\
%Nous allons donc, après un bref rappel des notations utilisées, commencer ce chapitre en redonnant l'argument initial de \textsc{Kempe} et montrer à quel endroit le bas blesse. Nous complèterons ensuite la démonstration par un raisonnement original qui permet d'achever la preuve de manière générale, sans l'utilisation d'outil informatique.\medskip
\begin{changemargin}{0cm}{0cm}
Ce chapitre va aborder successivement les points suivant:
\end{changemargin}
\begin{description}

\item[\textbf{La « démonstration » de Kempe:}] décrit l'argumentation initial de Kempe et montre à quel endroit "le bat blesse"
L'analyse de l’exception de Heawood nous permettra de définir avec précision la seule exception a la démonstration de Kempe; 
\item[\textbf{Les schémas de Soifer et Fritsch:}] nous permettra de nous confronter à 2 graphes minimaux couramment cités dans la problématique de la démonstration du théorème des 4 couleurs;
\item[\textbf{La démonstration:}] qui exposera avec précision la manière dont on peut démontrer le théorème.\\
\end{description}


%Nous allons donc, après un bref rappel des notations utilisées, plonger dans le vif du sujet:
%\section{Notations}

Rappelons brièvement nos définitions et notations. Nous appelons \textit{graphe géographique} tout graphe simple, planaire et connexe ; nous nous restreignis à ces hypothèses. Une \textit{chaine de \textsc{Kempe}} est un sous-graphe connexe maximum dichromatique. Nous notons $\mathcal{C}_{ab}$ la chaine comportant le sommet $a$ et dont la seconde couleur est celle du sommet $b$, celui-ci ne faisant pas nécessairement partie de la chaine. Lorsque la chaine $\mathcal{C}_{ab}$ contient à la fois le sommet~$a$ et le sommet $b$ (\textit{i.e.} lorsque~$\mathcal{C}_{ab}=\mathcal{C}_{ba}$), nous disons que la chaine $a$-$b$ est \textit{complète}, ce que l'on note~$\overline{\mathcal{C}}_{ab}$.

\section{« Démonstration » de Kempe}\label{sec:Kempe}
\subsection{l'argumentation initial de Kempe}

Soit un graphe géographique de $n>4\in\mathbb{N}^+$ sommets. On retire, un à un dans un ordre arbitraire, tous les sommets du graphe de degrés strictement inférieurs à six et ce jusqu'à n'avoir plus que quatre sommets\footnote{Si jamais la suppression d'un sommet divise le graphe en sous-graphes non connexes (au maximum cinq), on traite chacun des graphes obtenus séparément.}. Puisqu'à chaque fois que l'on supprime un sommet de degré $d$ on supprime également les $d$ arêtes qui le connectent au reste du graphe, chaque suppression d'un sommet tend à diminuer les degrés de ses voisins : ces derniers finiront tous, par le résultat démontré à la section~\ref{Existence}, tôt ou tard, par avoir un degré inférieur à six et on pourra toujours descendre jusqu'aux quatre sommets finaux. La procédure est donc complètement générale. Une fois que l'on n'a plus que quatre sommets, le graphe est clairement 4-coloriable : on choisi arbitrairement une couleur par sommet.

L'idée de la démonstration est ensuite de réintroduire, un par un et dans l'ordre inverse, chacun des sommets supprimés et de montrer qu'à chaque étape, le graphe est toujours 4-coloriable et ce indépendamment du nombre $n$ initial de sommets. Remarquons que, puisque au moment de leur suppression chaque sommet était de degré inférieur à six, chaque sommet que l'on réinsère est toujours de degré au plus cinq et on distingue trois cas.

\subsection*{Premier cas}

Le premier cas est le plus simple : si le degré du sommet inséré est plus petit ou égal à trois, sa coloration est triviale puisqu'il reste toujours au moins une couleur disponible pour colorer ce sommet.

\subsection*{Deuxième cas}

Dans le second cas, on considère les sommets de degré égal à quatre. Si deux (ou plus) des sommets voisins au sommet que l'on vient de réinséré sont colorés de la même couleur, il reste à nouveau au moins une couleur disponible et le cas est trivial.

En revanche, si les quatre sommets voisins sont coloré à l'aide des quatre couleurs différentes, on opère de manière analogue à la démonstration du théorème des cinq couleurs. On dénote «~0~» le sommet réinséré que l'on souhaite colorer et on numérote de 1 à 4 ses sommets voisins\footnote{À nouveau, on suppose de plus que ces sommets forment un chemin fermé autour de 0, reliées par les arrêtes 1-2, 2-3, 3-4 et 4-1.} (voir Figure~\ref{fig:Deuxième}). Si la chaine $\mathcal{C}_{13}$ n'est pas complète (\textit{i.e.}~$\overline{\mathcal{C}}_{13}\,\nexists$), on peut permuter les couleurs dans la chaine $\mathcal{C}_{31}$ sans modifier le sommet 1 : les sommets 1 et 3 auront la même couleur finale et la seconde couleur de la chaine sera disponible pour le sommet 0\footnote{Le choix du sommet sur lequel s'effectue la permutation de couleur, c'est-à-dire le choix de la chaine $\mathcal{C}_{13}$ ou de $\mathcal{C}_{31}$, est arbitraire et on peut, de manière équivalente, permuter les couleurs dans la chaine $\mathcal{C}_{13}$. De même, le choix de numérotation des sommets étant tout autant arbitraire, on peut s'intéresser de manière équivalente à la chaine $\mathcal{C}_{24}$.}.

\begin{SCfigure}[][h!]
	\begin{tikzpicture}
\def \n {4}
\def \a {90}
\def \x {360/\n * (\s - 1)+\a}
\def \radius {2cm}
\def \m {7}
\def \y {360/\m*(\t-1)+\c}
\def \radium {.6cm}
\node[draw, circle,fill=w] (0) at (0,0) {0};
\draw[dashed] (0,2) arc (70:290:2.125cm);
\def \s {1} \node[draw, circle,fill=v]	(\s)  at ({\x}:\radius) {$\s$}; \path (\s) edge (0); \def \c {-90}
	\def \t {3} \coordinate[shift=({\x}:\radius)] (c\t) at ({\y}:\radium); \path (\s) edge (c\t);
	\def \t {4} \coordinate[shift=({\x}:\radius)] (c\t) at ({\y}:\radium); \path (\s) edge (c\t);
	\def \t {5} \coordinate[shift=({\x}:\radius)] (c\t) at ({\y}:\radium); \path (\s) edge (c\t);
%	\def \t {6} \coordinate[shift=({\x}:\radius)] (c\t) at ({\y}:\radium); \path (\s) edge (c\t);
\def \s {2} \node[draw, circle,fill=r]	(\s)  at ({\x}:\radius) {$\s$}; \path (\s) edge (0); \path (\s) edge (1); \def \c {0}
	\def \t {3} \coordinate[shift=({\x}:\radius)] (c\t) at ({\y}:\radium); \path (\s) edge (c\t);
	\def \t {4} \coordinate[shift=({\x}:\radius)] (c\t) at ({\y}:\radium); \path (\s) edge (c\t);
	\def \t {5} \coordinate[shift=({\x}:\radius)] (c\t) at ({\y}:\radium); \path (\s) edge (c\t);
	\def \t {6} \coordinate[shift=({\x}:\radius)] (c\t) at ({\y}:\radium); \path (\s) edge (c\t);
\def \s {3} \node[draw, circle,fill=j]	(\s)  at ({\x}:\radius) {$\s$}; \path (\s) edge (0); \path (\s) edge (2); \def \c {90}
%	\def \t {3} \coordinate[shift=({\x}:\radius)] (c\t) at ({\y}:\radium); \path (\s) edge (c\t);
	\def \t {4} \coordinate[shift=({\x}:\radius)] (c\t) at ({\y}:\radium); \path (\s) edge (c\t);
	\def \t {5} \coordinate[shift=({\x}:\radius)] (c\t) at ({\y}:\radium); \path (\s) edge (c\t);
	\def \t {6} \coordinate[shift=({\x}:\radius)] (c\t) at ({\y}:\radium); \path (\s) edge (c\t);
\def \s {4} \node[draw, circle,fill=b]	(\s)  at ({\x}:\radius) {$\s$}; \path (\s) edge (0); \path (\s) edge (3); \path (\s) edge (1); \def \c {180}
	\def \t {3} \coordinate[shift=({\x}:\radius)] (c\t) at ({\y}:\radium); \path (\s) edge (c\t);
	\def \t {4} \coordinate[shift=({\x}:\radius)] (c\t) at ({\y}:\radium); \path (\s) edge (c\t);
	\def \t {5} \coordinate[shift=({\x}:\radius)] (c\t) at ({\y}:\radium); \path (\s) edge (c\t);
	\def \t {6} \coordinate[shift=({\x}:\radius)] (c\t) at ({\y}:\radium); \path (\s) edge (c\t);
%\node (-1) at (3,0) {};
\end{tikzpicture}
	\caption{Deuxième cas\\\footnotesize\textit{La chaine~$\overline{\mathcal{C}}_{13}$ existe-t-elle~?}}
	\label{fig:Deuxième}
\end{SCfigure}

Si la chaine~$\overline{\mathcal{C}}_{13}$ est complète, il ne sert à rien de permuter les couleurs dans cette chaine : les deux sommets ne feraient qu'échanger leurs couleurs. Mais dans ce cas, il ne peut pas exister de chaine complète~$\overline{\mathcal{C}}_{24}$ puisqu'une telle chaine devrait croiser~$\overline{\mathcal{C}}_{13}$ en un sommet de couleur commune, or celles-ci sont de couleurs différentes. On est dès lors libre de permuter les couleurs dans~$\mathcal{C}_{24}$ ou~$\mathcal{C}_{42}$ de façon à ce que 2 et 4 soient de la même couleur et qu'une couleur soit libérée pour colorer le sommet 0.

\subsection*{Troisième cas}

Enfin, le troisième cas~\label{pg:Troisième} s'occupe des sommets de degré cinq. Pour un tel sommet, au moins deux de ses voisins sont nécessairement de même couleurs et si plus de deux ont la même couleur, le cas est trivial : il reste au moins une couleur disponible pour le colorer.

Si seuls deux sommets sont homochromes, il existe un sommet dont les deux voisins ont la même couleurs\footnote{Encore une fois, on suppose que les sommets 1 à 5 sont liés en un chemin fermé autour de 0. Ceci est une hypothèse plus contraignante : s'ils n'étaient pas liés entre eux, les sommets voisins de 0 pourraient tous être de la même couleur. Puisque tout graphe 4-coloriable avec notre méthode est nécessairement 4-coloriable lorsqu'on lui supprime n'importe quelle arrête, cette hypothèse ne réduit pas la généralité de la démonstration.}. La situation est celle reprise sur le premier graphe de la Figure~\ref{fig:Troisième}, où les sommets 2 et 5 de même couleur sont séparés par le sommet 1. On regarde d'abord s'il existe une chaine complète~$\overline{\mathcal{C}}_{13}$ (resp.~$\overline{\mathcal{C}}_{14}$). Si une telle chaine complète n'existe pas, on peut permuter les couleurs dans $\mathcal{C}_{13}$ ou~$\mathcal{C}_{31}$ (resp. $\mathcal{C}_{14}$ ou $\mathcal{C}_{41}$) et de ce fait libérer la couleur de l'un de ces sommets pour le sommet 0.

Dans le cas où à la fois~$\overline{\mathcal{C}}_{13}$ et~$\overline{\mathcal{C}}_{14}$ existent, alors (par non croisement des chaines de couleurs différentes) les chaines~$\mathcal{C}_{24}$ et~$\mathcal{C}_{53}$ ne peuvent pas être complètes (\textit{i.e.} elles ne peuvent pas contenir, respectivement, les sommets~4 et~3). La preuve de \textsc{Kempe} s'achève en stipulant que l'on peut dès lors permuter les couleurs dans ces deux chaines, comme illustré sur le deuxième graphe de la Figure~\ref{fig:Troisième}, pour éliminer la couleur commune des sommets 2 et 5, qui peut ensuite être utilisée pour le sommet~0. A priori, la preuve est alors complète, tous les cas ayant été traités, et le théorème des quatre couleurs est démontré.

Mais il y a un \textit{mais} …

\begin{figure}[h!]\centering
\begin{changemargin}{-3cm}{-3cm}
\begin{center}
\begin{tikzpicture}
\def \n {5}
\def \a {90}
\def \x {360/\n * (\s - 1)+\a}
\def \radius {2cm}
\def \m {7}
\def \y {360/\m*(\t-1)+\c}
\def \radium {.6cm}
\node[draw, circle,fill=w] (0) at (0,0) {0};
\def \s {1} \node[draw, circle,fill=r]	(\s)  at ({\x}:\radius) {$\s$}; \path (\s) edge (0); \def \c {-90}
%	\def \t {3} \coordinate[shift=({\x}:\radius)] (c\t) at ({\y}:\radium); \path (\s) edge (c\t);
	\def \t {4} \coordinate[shift=({\x}:\radius)] (c\t) at ({\y}:\radium); \path (\s) edge (c\t);
	\def \t {5} \coordinate[shift=({\x}:\radius)] (c\t) at ({\y}:\radium); \path (\s) edge (c\t);
%	\def \t {6} \coordinate[shift=({\x}:\radius)] (c\t) at ({\y}:\radium); \path (\s) edge (c\t);
\def \s {2} \node[draw, circle,fill=b]	(\s)  at ({\x}:\radius) {$\s$}; \path (\s) edge (0); \def \c {-18} 	\path (\s) edge (1);
	\def \t {3} \coordinate[shift=({\x}:\radius)] (c\t) at ({\y}:\radium); \path (\s) edge (c\t);
	\def \t {4} \coordinate[shift=({\x}:\radius)] (c\t) at ({\y}:\radium); \path (\s) edge (c\t);
	\def \t {5} \coordinate[shift=({\x}:\radius)] (c\t) at ({\y}:\radium); \path (\s) edge (c\t);
	\def \t {6} \coordinate[shift=({\x}:\radius)] (c\t) at ({\y}:\radium); \path (\s) edge (c\t);
\def \s {3} \node[draw, circle,fill=v]	(\s)  at ({\x}:\radius) {$\s$}; \path (\s) edge (0); \def \c {54}	\path (\s) edge (2);
%	\def \t {3} \coordinate[shift=({\x}:\radius)] (c\t) at ({\y}:\radium); \path (\s) edge (c\t);
	\def \t {4} \coordinate[shift=({\x}:\radius)] (c\t) at ({\y}:\radium); \path (\s) edge (c\t);
	\def \t {5} \coordinate[shift=({\x}:\radius)] (c\t) at ({\y}:\radium); \path (\s) edge (c\t);
	\def \t {6} \coordinate[shift=({\x}:\radius)] (c\t) at ({\y}:\radium); \path (\s) edge (c\t);
\def \s {4} \node[draw, circle,fill=j]	(\s)  at ({\x}:\radius) {$\s$}; \path (\s) edge (0); \def \c {126}	\path (\s) edge (3);
	\def \t {3} \coordinate[shift=({\x}:\radius)] (c\t) at ({\y}:\radium); \path (\s) edge (c\t);
	\def \t {4} \coordinate[shift=({\x}:\radius)] (c\t) at ({\y}:\radium); \path (\s) edge (c\t);
	\def \t {5} \coordinate[shift=({\x}:\radius)] (c\t) at ({\y}:\radium); \path (\s) edge (c\t);
%	\def \t {6} \coordinate[shift=({\x}:\radius)] (c\t) at ({\y}:\radium); \path (\s) edge (c\t);
\def \s {5} \node[draw, circle,fill=b]	(\s)  at ({\x}:\radius) {$\s$}; \path (\s) edge (0); \def \c {198}	\path (\s) edge (4); \path (\s) edge (1);
	\def \t {3} \coordinate[shift=({\x}:\radius)] (c\t) at ({\y}:\radium); \path (\s) edge (c\t);
	\def \t {4} \coordinate[shift=({\x}:\radius)] (c\t) at ({\y}:\radium); \path (\s) edge (c\t);
	\def \t {5} \coordinate[shift=({\x}:\radius)] (c\t) at ({\y}:\radium); \path (\s) edge (c\t);
	\def \t {6} \coordinate[shift=({\x}:\radius)] (c\t) at ({\y}:\radium); \path (\s) edge (c\t);
\path (3) edge[looseness=1.8,bend left=90,dashed] (1);
\path (4) edge[looseness=1.8,bend right=90,dashed] (1);
\end{tikzpicture}
\input{./graphe/4c_5_sommets_3col}
\caption{Troisième cas\\ \footnotesize\textit{Si~$\overline{\mathcal{C}}_{13}$ et~$\overline{\mathcal{C}}_{14}$ existent, \textsc{Kempe} nous dit qu'on peut permuter les couleurs dans $\mathcal{C}_{24}$ et $\mathcal{C}_{53}$.}}\label{fig:Troisième}
\end{center}
\end{changemargin}
\end{figure}

\subsection{Un contre-exemple}

Le problème dans la démonstration de \textsc{Kempe} se situe dans la dernière phrase :
\begin{changemargin}{.4cm}{.4cm}
	[…] \textit{on peut dès lors permuter les couleurs dans ces deux chaines, comme illustré sur le deuxième graphe de la Figure~\ref{fig:Troisième}, pour éliminer la couleur commune des sommets 2 et 5} […]
\end{changemargin}
En fait, lorsque l'on permute les couleurs des deux chaines, on est bien obligé de le faire d'abord dans l'une, puis dans l'autre. Et il existe des cas où l'enchainement de ces permutations crée de nouvelles chaines qui modifient les couleurs des sommets autours de 0 et empêchent d'éliminer la couleur désirée. Mais mis à part le traitement de ce troisième et dernier cas, le reste de la démonstration est absolument correct !

Le premier contre-exemple a été donné par \textsc{Heawood} en 1890. D'autres contre-exemples ont été donnés dans la suite, dont deux graphes minimaux.  Prenons l'exemple du graphe minimal de \textsc{Fritsch}, donnée en 1998, repris à la Figure~\ref{fig:Fritsch}, où il ne reste plus qu'un sommet à colorer (que l'on note 0, comme précédemment).  On retrouve le troisième cas de la Figure~\ref{fig:Troisième}, entouré de trois sommets supplémentaires qui forment deux chaines complètes~$\overline{\mathcal{C}}_{13}$ et~$\overline{\mathcal{C}}_{14}$, que l'on a colorées pour plus de visibilité. On remarque que ces deux chaines se croisent sur le sommet 6, de la couleur commune.

\begin{SCfigure}[][t!]
	\input{./graphe/Fritsch_2Chaine}
	\caption{Graphe de Fritsch\\\footnotesize\textit{Contre-exemple minimal} [1998]}\label{fig:Fritsch}
\end{SCfigure}
\FloatBarrier

Si on suit l'argument de la démonstration de \textsc{Kempe}, on peut alors permuter les couleurs de~$\overline{\mathcal{C}}_{24}$ et de~$\mathcal{C}_{53}$ pour se débarrasser de la couleur commune à~2 et 5. On voit cependant à la Figure~\ref{fig:Fritsch_Inversion} (a gauche) que l'inversion des couleurs dans~$\overline{\mathcal{C}}_{24}$ brise la chaine 
$\overline{\mathcal{C}}_{14}$ et complète la chaine~$\overline{\mathcal{C}}_{53}$ (qui devient donc~$\overline{\mathcal{C}}_{53}$ puisqu'alors~$\overline{\mathcal{C}}_{35}$=$\mathcal{C}_{53}$) : permuter les couleurs de la chaine~$\overline{\mathcal{C}}_{53}$ ne permettra donc pas de libérer une couleur~(Figure~\ref{fig:Fritsch_Inversion} à droite). Après avoir appliqué la méthode proposée par \textsc{Kempe}, on se retrouve donc toujours avec quatre couleurs autours du sommet~0, dans une configuration de type du troisième cas (voir Figure~\ref{fig:Troisième}).

\begin{figure}[!h]\centering
	\begin{changemargin}{-3cm}{-3cm}
		\begin{center}
			%\begin{tikzpicture}
%	%line width=5pt pour les ligne droite
%	\def \n {5}
%	\def \a {90}
%	\def \x {360/\n * (\s - 1)+\a}
%	\def \radius {1.1cm}
%	\def \radium {2.2cm}
%	\node[draw,circle,fill=w] (0) at (0,0) {0};
%	% les 5 noeud interieur
%	\def \s {1} \node[draw,circle,fill=r]	(\s)  at ({\x}:\radius)	{\s};	\path (\s) edge (0);
%	\def \s {2} \node[draw,circle,fill=j]	(\s)  at ({\x}:\radius)	{\s};	\path (\s) edge (0);	\path (\s) edge (1);
%	\def \s {3} \node[draw,circle,fill=v]	(\s)  at ({\x}:\radius)	{\s};	\path (\s) edge (0);	\path (\s) edge (2);
%	\def \s {4} \node[draw,circle,fill=j]	(\s)  at ({\x}:\radius)	{\s};	\path (\s) edge (0);	\path (\s) edge (3);
%	\def \s {5} \node[draw,circle,fill=b]	(\s)  at ({\x}:\radius)	{\s};	\path (\s) edge (0);	\path (\s) edge (4);	\path (\s) edge (1);
%	% les 3 arc exterieur
%	\draw [bicolor={v}{b},line width=5pt] (0,0) ++(168:\radium) arc (168:18:\radium);
%	\draw (0,0) ++(18:\radium) arc (18:-90:\radium);
%	%arc en 2 morceau...
%	\draw (0,0) ++(-90:\radium) arc (-90:-180:\radium);
%	\draw (0,0) ++(180:\radium) arc (180:168:\radium);
%	% les 3 noeds exterieur	
%	\def \t {1} \node[draw,circle,fill=b]	(6)   at (162:\radium) {6};		
%	\def \t {2} \node[draw,circle,fill=r]	(7)   at (270:\radium) 	{7};		
%	\def \t {3} \node[draw,circle,fill=v]	(8)   at (18:\radium) 	{8};		
%	\path (1) edge (6);				
%	\path (2) edge (6);		
%	\path (3) edge[bicolor={v}{b},line width=5pt] (6);
%	\path (3) edge (7);			
%	\path (4) edge (7); 			
%	\path (1) edge (8);			
%	\path (4) edge (8);		
%	\path  (5) edge[bicolor={v}{b},line width=5pt] (8);		
%	%\node[draw] at (\radium,\radium) {permutation 1-3};
%\end{tikzpicture}
\begin{tikzpicture}[scale=.75]
\def \n {5}
\def \a {90}
\def \x {360/\n * (\s - 1)+\a}
\def \radius {2cm}
\def \radium {3.618cm}
\node[draw,circle,fill=w] (0) at (0,0) {0};
\def \s {1} \node[draw,circle,fill=r]	(\s)  at ({\x}:\radius)	{\s};	\path (\s) edge (0);
\def \s {2} \node[draw,circle,fill=j]	(\s)  at ({\x}:\radius)	{\s};	\path (\s) edge (0);	\path (\s) edge (1);
\def \s {3} \node[draw,circle,fill=v]	(\s)  at ({\x}:\radius)	{\s};	\path (\s) edge (0);	\path (\s) edge (2);
\def \s {4} \node[draw,circle,fill=j]		(\s)  at ({\x}:\radius)	{\s};	\path (\s) edge (0);	\path (\s) edge (3);
\def \s {5} \node[draw,circle,fill=b]	(\s) 	at ({\x}:\radius)	{\s};	\path (\s) edge (0);	\path (\s) edge (4);	\path (\s) edge (1);
\def \t {1} \node[draw,circle,fill=b]		(6) 	at (162:\radium) {7};		\path (1) edge (6);		\path (2) edge (6);		\path (3) edge[bicolor={b}{v},line width=2pt] (6);
\def \t {2} \node[draw,circle,fill=r]		(7) 	at (270:3.236) 	{6};		\path (3) edge[bicolor={r}{v},line width=2pt] (7);		\path (4) edge (7);
			\path (6) edge[bend right] (7);
\def \t {3} \node[draw,circle,fill=v]	(8) 	at (18:\radium) 	{8};		\path (1) edge[bicolor={r}{v},line width=2pt] (8);		\path (4) edge (8);			\path (5) edge[bicolor={b}{v},line width=2pt] (8);	
			\path (6) edge[looseness=1.1, bend left=50,bicolor={b}{v},line width=2pt] (8);					\path (7) edge[bend right,bicolor={r}{v},line width=2pt] (8);
\end{tikzpicture}
			\hspace{15pt}
			\input{./graphe/Fritsch_Err}
		\end{center}
	\end{changemargin}
	\begin{changemargin}{1.75cm}{1.75cm}
		\caption{Inversion de couleurs\\ \footnotesize\textit{À partir du graphe de \textsc{Fritsch} , inverser les couleurs dans~$\overline{\mathcal{C}}_{24}$ introduit une chaine complète~$\overline{\mathcal{C}}_{53}$ (a gauche). Permuter ensuite les couleurs dans cette chaine n'apporte dès lors pas la simplification désirée (à droite).}}\label{fig:Fritsch_Inversion}
	\end{changemargin}	
\end{figure}
\FloatBarrier	


Cependant, cela ne veut pas dire qu'il est impossible de colorer ce graphe avec seulement quatre couleurs (l'idée est justement de montrer dans ce document que tout graphe géographique est toujours 4-coloriable!). La Figure~\ref{fig:Fritsch_ok} illustre une quadricoloration valide du graphe.

\begin{SCfigure}[][!ht]
	\input{./graphe/Fritsch_ok}
	\caption{\\Exemple de coloration% du grape de Fritsch
	}\label{fig:Fritsch_ok}
\end{SCfigure}
\FloatBarrier

\subsection{Les limites de la méthode de Kempe}

L'existence de conte-exemples prouve que la démonstration de \textsc{Kempe} est incorrecte, en particulier le traitement du troisième cas possédant deux chaines~$\overline{\mathcal{C}}_{13}=\mathcal{C}_{31}$ et~$\overline{\mathcal{C}}_{14}=\mathcal{C}_{41}$. Lorsque ces deux chaines existent et qu'elles se croisent, la démonstration ne prend pas en compte le fait que l'inversion de la première chaine peut briser la seconde, ce qui engendre une modification des hypothèses de départ (lesquelles stipulent qu'il existe deux chaines) : la méthode n'est donc plus applicable.\\

\subsection{Analyse des conditions minimum de l'exception de \textsc{Heawood}}
On peut définir le graphe de différentes manières :
\begin{description}
\item [par négation] : tous les graphes qui ne peuvent être résolus par les méthodes de \textsc{Kempe}.
Il faut rappeler que notre but étant de trouver une solution (et pas de mettre la méthode de \textsc{Kempe} « partiellement » en défaut), nous ne nous intéressons qu'au cas où les 2 permutations des chaines de \textsc{Kempe} sont mises en échec. Si $\mathcal{PC}_{24}$ puis $\mathcal{PC}_{53}$ ne fonctionne pas (il persiste 4 couleurs pour les voisins de 0), mais que $\mathcal{PC}_{53}$ puis $\mathcal{PC}_{24}$ fonctionne, nous considérerons le problème comme ayant une solution (et donc ne demandant pas de traitement complémentaire). Dans la suite du document, nous vérifions systématiquement que les 2 ordres de permutation sont bien pris en compte.\\
\item[par ces contraintes]
En définissant les conditions auquel le schema doit obligatoirement répondre pour etre dans ce cas de figure.
\end{description}
\paragraph{1er condition}
Nous sommes dans le cas où $\overline{\mathcal{C}}_{13}$ et $\overline{\mathcal{C}}_{14}$ existe.
\begin{figure}[!ht]\centering
	\begin{changemargin}{-3cm}{-3cm}
		\begin{center}
			\begin{tikzpicture}
	\def \n {5}
	\def \a {90}
	\def \x {360/\n * (\s - 1)+\a}
	\def \radius {2cm}
	\def \m {7}
	\def \y {360/\m*(\t-1)+\c}
	\def \radium {.6cm}
	\node[draw, circle,fill=w] (0) at (0,0) {0};
	\def \s {1} \node[draw, circle,fill=r]	(\s)  at ({\x}:\radius) {$\s$}; \path (\s) edge (0); \def \c {-72}
	\def \s {2} \node[draw, circle,fill=b]	(\s)  at ({\x}:\radius) {$\s$}; \path (\s) edge (0); \def \c {0} 	\path (\s) edge (1);
	\def \s {3} \node[draw, circle,fill=v]	(\s)  at ({\x}:\radius) {$\s$}; \path (\s) edge (0); \def \c {72}	\path (\s) edge (2);
	\def \s {4} \node[draw, circle,fill=j]	(\s)  at ({\x}:\radius) {$\s$}; \path (\s) edge (0); \def \c {144}	\path (\s) edge (3);
	\def \s {5} \node[draw, circle,fill=b]	(\s)  at ({\x}:\radius) {$\s$}; \path (\s) edge (0); \def \c {216}	\path (\s) edge (4);
	\path (1) edge (5);
	\path (3) edge[looseness=1.8,bend left=90,bicolor={v}{r},line width=4pt] (1);
	\path (4) edge[looseness=1.8,bend right=90,bicolor={j}{r},line width=4pt] (1);
\end{tikzpicture}
			\caption{$\overline{\mathcal{C}}_{13}$ et $\overline{\mathcal{C}}_{14}$ existe}\label{fig:condition1}
		\end{center}
	\end{changemargin}
\end{figure}
\FloatBarrier

\paragraph{2em condition}
Il doit exister au moins un sommet commun à la $\overline{\mathcal{C}}_{14}$ et à la $\mathcal{C}_{24}$. Et en symétrie au moins un sommet commun à la $\overline{\mathcal{C}}_{13}$ et a la $\mathcal{C}_{53}$.\\
De manière plus synthétique on a :\\
$\overline{\mathcal{C}}_{14}\cap\mathcal{C}_{24}\not=\varnothing$\\
$\overline{\mathcal{C}}_{12}\cap\mathcal{C}_{53}\not=\varnothing$\\
Lorsque les deux chaines $\overline{\mathcal{C}}_{13}$ et $\overline{\mathcal{C}}_{14}$ ne se croisent pas (figure~\ref{fig:cas4_cond2_1}), les sommets 2 et 5 sont respectivement enclavés par des chaines ne contenant pas les couleurs de $\mathcal{C}_{24}$ et $\mathcal{C}_{53}$. Sur la figure, le sommet 2 est enclavé par une chaine qui n’est ni bleue ni jaune et le sommet 5 par une chaine qui n’est ni bleue ni verte. L’inversion des couleurs dans $\mathcal{C}_{24}$ et $\mathcal{C}_{53}$ n’a alors d’influence que strictement à l’intérieur de ces enclaves et la méthode de \textsc{Kempe} fonctionne toujours.\\

\begin{figure}[!ht]\centering
	\begin{changemargin}{-3cm}{-3cm}
		\begin{center}
			\begin{tikzpicture}
	%line width=5pt pour les ligne droite
	\def \n {5}
	\def \a {90}
	\def \x {360/\n * (\s - 1)+\a}
	\def \radius {1.5cm}
	\def \radium {2.2cm}
	\node[draw,circle,fill=w] (0) at (0,0) {0};
	% les 5 noeud interieur
	\def \s {1} \node[draw,circle,fill=r]	(\s)  at ({\x}:\radius)	{\s};	\path (\s) edge (0);
	\def \s {2} \node[draw,circle,fill=b]	(\s)  at ({\x}:\radius)	{\s};	\path (\s) edge (0);	\path (\s) edge (1);
	\def \s {3} \node[draw,circle,fill=v]	(\s)  at ({\x}:\radius)	{\s};	\path (\s) edge (0);	\path (\s) edge (2);
	\def \s {4} \node[draw,circle,fill=j]	(\s)  at ({\x}:\radius)	{\s};	\path (\s) edge (0);	\path (\s) edge (3);
	\def \s {5} \node[draw,circle,fill=b]	(\s)  at ({\x}:\radius)	{\s};	\path (\s) edge (0);	\path (\s) edge (4);
	\path (\s) edge (1);
	%
	\path (3) edge[looseness=1.8,bend left=90,bicolor={v}{r},line width=4pt] (1);
	\path (4) edge[looseness=1.8,bend right=90,bicolor={j}{r},line width=4pt] (1);
	
	%
	%\path (1) edge[looseness=3,bend left=-100,bicolor={v}{r},line width=4pt] (3);
	%\path (1) edge[looseness=4,bend right=120,bend right=120,bicolor={j}{r},line width=4pt] (4);
	%
	\draw[bicolor={j}{b},line width=2pt] (2) -- (-1.5,1.5);
	\draw[bicolor={v}{b},line width=2pt] (5) -- ( 1.5,1.5);
\end{tikzpicture}
			\caption{$\overline{\mathcal{C}}_{13}$ et $\overline{\mathcal{C}}_{14}$ ne se croise pas}\label{fig:cas4_cond2_1}
		\end{center}
	\end{changemargin}
\end{figure}
\FloatBarrier
En revanche, lorsque les chaines $\overline{\mathcal{C}}_{13}$ et $\overline{\mathcal{C}}_{14}$ se croisent (figure~\ref{fig:cas4_cond2_2} ), elles ne forment plus d’enclave perméable autour des sommets 5 et 2, puisqu’elles contiennent respectivement la deuxième couleur des chaines $\overline{\mathcal{C}}_{53}$ et $\overline{\mathcal{C}}_{24}$. Ainsi, il peut exister un croisement entre $\overline{\mathcal{C}}_{24}$ et $\overline{\mathcal{C}}_{14}$ (resp. entre $\overline{\mathcal{C}}_{53}$ et $\overline{\mathcal{C}}_{13}$) puisqu’elles partagent la couleur du sommet 4 (resp. du sommet 3) et la permutation de couleur dans $\overline{\mathcal{C}}_{24}$ (resp. $\overline{\mathcal{C}}_{53}$) peut passer à travers la chaine $\overline{\mathcal{C}}_{14}$ (resp. $\overline{\mathcal{C}}_{13}$). À l’endroit précis d’un tel croisement (s’il existe), la chaine $\overline{\mathcal{C}}_{14}$ (resp. $\overline{\mathcal{C}}_{13}$) va être :
brisée par la permutation de couleurs dans $\overline{\mathcal{C}}_{24}$ (resp. $\overline{\mathcal{C}}_{53}$) et le raisonnement de \textsc{Kempe} n’est plus applicable.\\
\\
3 nouveau sommet sont nessesaire pour attindre cette condition, nous les appelons 6,7 et 8.

\begin{figure}[!ht]\centering
	\begin{changemargin}{-3cm}{-3cm}
		\begin{center}
			\begin{tikzpicture}
	%line width=5pt pour les ligne droite
	\def \n {5}
	\def \a {90}
	\def \x {360/\n * (\s - 1)+\a}
	\def \radius {1.5cm}
	\def \radium {2.2cm}
	\node[draw,circle,fill=w] (0) at (0,0) {0};
	% les 5 noeud interieur
	\def \s {1} \node[draw,circle,fill=r]	(\s)  at ({\x}:\radius)	{\s};	\path (\s) edge (0);
	\def \s {2} \node[draw,circle,fill=b]	(\s)  at ({\x}:\radius)	{\s};	\path (\s) edge (0);	\path (\s) edge (1);
	\def \s {3} \node[draw,circle,fill=v]	(\s)  at ({\x}:\radius)	{\s};	\path (\s) edge (0);	\path (\s) edge (2);
	\def \s {4} \node[draw,circle,fill=j]	(\s)  at ({\x}:\radius)	{\s};	\path (\s) edge (0);	\path (\s) edge (3);
	\def \s {5} \node[draw,circle,fill=b]	(\s)  at ({\x}:\radius)	{\s};	\path (\s) edge (0);	\path (\s) edge (4);
	\path (\s) edge (1);
	%
	%\path (3) edge[looseness=1.8,bend left=90,bicolor={v}{r},line width=4pt] (1);
	\path (1) edge[looseness=3,bend left=-100,bicolor={v}{r},line width=4pt] (3);
	\path (1) edge[looseness=4,bend right=120,bicolor={j}{r},line width=4pt] (4);
	%
	\draw[bicolor={j}{b},line width=2pt] (2) -- (-2.9,0.7);
	\path (5) edge[looseness=2,bend left=-60,bicolor={v}{b},line width=3pt] (-1.5,2);
	\node[draw,circle,fill=j]	()  at (-2.8,0.7)	{};
	\node[draw,circle,fill=v]	()  at (-1.4,2)	{};
	\node[draw,circle,fill=r]	()  at (-3.0,-0.1)	{};
	
\end{tikzpicture}
			\caption{$\overline{\mathcal{C}}_{13}$ et $\overline{\mathcal{C}}_{14}$ se croise}\label{fig:cas4_cond2_2}
		\end{center}
	\end{changemargin}
\end{figure}
\FloatBarrier
Les 2 chaînes $\mathcal{C}_{24}$ et $\mathcal{C}_{53}$ peuvent avoir une intersection avec respectivement la  $\overline{\mathcal{C}}_{13}$  et la  $\overline{\mathcal{C}}_{14}$\\
La figure~\ref{fig:cas4_cond2} (de droite) est identique a la figure~\ref{fig:cas4_cond2_2} mais représenté de manière symétrique\\

\begin{figure}[!h]\centering
	\begin{changemargin}{-3cm}{-3cm}
		\begin{center}
			\input{./graphe/cas4_cond2_3}
			\hspace{15pt}
			\begin{tikzpicture}
	%line width=5pt pour les ligne droite
	\def \n {5}
	\def \a {90}
	\def \x {360/\n * (\s - 1)+\a}
	\def \radius {1.1cm}
	\def \radium {2.2cm}
	\node[draw,circle,fill=w] (0) at (0,0) {0};
	% les 5 noeud interieur
	\def \s {1} \node[draw,circle,fill=r]	(\s)  at ({\x}:\radius)	{\s};	\path (\s) edge (0);
	\def \s {2} \node[draw,circle,fill=b]	(\s)  at ({\x}:\radius)	{\s};	\path (\s) edge (0);	\path (\s) edge (1);
	\def \s {3} \node[draw,circle,fill=v]	(\s)  at ({\x}:\radius)	{\s};	\path (\s) edge (0);	\path (\s) edge (2);
	\def \s {4} \node[draw,circle,fill=j]	(\s)  at ({\x}:\radius)	{\s};	\path (\s) edge (0);	\path (\s) edge (3);
	\def \s {5} \node[draw,circle,fill=b]	(\s)  at ({\x}:\radius)	{\s};	\path (\s) edge (0);	\path (\s) edge (4);	\path (\s) edge (1);
	% les 3 arc exterieur
	%\draw (0,0) ++(168:\radium) arc (168:18:\radium);
	\draw[bicolor={v}{r},line width=4pt] (0,0) ++(18:\radium) arc (18:-90:\radium);
	%arc en 2 morceau...
	%\draw (0,0) ++(-90:\radium) arc (-90:-180:\radium);
	%\draw (0,0) ++(180:\radium) arc (180:168:\radium);
	% les 3 noeds exterieur	
	%\node[draw,circle,fill=j]	(6)   at (162:\radium) {6};		
	\node[draw,circle,fill=r]	(7)   at (270:\radium) 	{7};		
	\node[draw,circle,fill=v]	(8)   at (18:\radium) 	{8};		
	%\path (1) edge (6);				
	%\path (2) edge (6);
	%\path (3) edge (6);
	\path (3) edge[bicolor={v}{r},line width=4pt] (7);			
	%\path (4) edge (7); 			
	\path (1) edge[bicolor={v}{r},line width=4pt] (8);			
	%\path (4) edge (8);		
	\path (5) edge[bicolor={v}{b},line width=3pt] (8);		
	%\node[draw] at (\radium,\radium) {permutation 1-3};
\end{tikzpicture}
			\hspace{15pt}
			\input{./graphe/cas4_cond2_5}
		\end{center}
	\end{changemargin}
	\begin{changemargin}{1.75cm}{1.75cm}
		\caption{Condition 2}\label{fig:cas4_cond2}
	\end{changemargin}
\end{figure}
\FloatBarrier


\paragraph{3em condition}
Pour être dans l'exception de \textsc{Heawood} et que les méthodes de \textsc{Kempe} soient mises en échec, il faut que la modification, qui a lieux dans la chaîne  $\overline{\mathcal{C}}_{13}$  (décrite en condition 2) lors de la  $\mathcal{PC}_{53}$, engendre une  $\overline{\mathcal{C}}_{24}$ qui aura pour conséquence de ne pas pouvoir réduire les couleurs voisines du sommet 0 à 3.
Pour permettre cette chaîne, on ajoute une arrête entre les sommets 6 et 8 au schéma de la figure~\ref{fig:cas4_cond2}
\begin{figure}[!ht]\centering
	\input{./graphe/cas4_cond3_1}	
	\caption{}\label{fig:cas4_cond3_1}
\end{figure}
\FloatBarrier
La même condition en symétrie doit être vrai ; la  $\mathcal{PC}_{24}$ modifie la  $\overline{\mathcal{C}}_{14}$ et crée une  $\overline{\mathcal{C}}_{53}$ qui empêche (lors de la $\mathcal{PC}_{24}$) de réduire les couleurs autour du sommet 0 à 3.\\
Elle emprunte la même chaine  $\overline{\mathcal{C}}_{68}$ (figure~\ref{fig:cas4_cond3_2}).
\begin{figure}[!ht]\centering
	\input{./graphe/cas4_cond3_2}	
	\caption{}\label{fig:cas4_cond3_2}
\end{figure}
\FloatBarrier
Cela nous donne (Figure~\ref{fig:cas4_cond3_3}) le schéma \textbf{presque} minimal des conditions de l'exception d'\textsc{Heawood}. 
\begin{figure}[!ht]\centering
	\input{./graphe/cas4_cond3_3}	
	\caption{}\label{fig:cas4_cond3_3}
\end{figure}
\FloatBarrier
Presque, parce qu'au moment de l'ajout de l’arête 6-8 pour satisfaire à la 3eme condition de l'exception d'\textsc{Heawood}, on crée un dédoublement de la  $\mathcal{C}_{34}$. L'une comprend uniquement les sommets 3 et 4, l'autre comprend les sommets 3, 6, 8, 4.
Ce dédoublement n'apporte rien et sans l’arête 3, 4, notre schéma minimal répond toujours à toutes les conditions de l'exception d'\textsc{Heawood}.
Ce qui nous permet de représenter le schéma minimal des conditions de l'exception d'\textsc{Heawood} ci-dessous (figure~\ref{fig:cas4_cond3_4})
\begin{figure}[!ht]\centering
	\input{./graphe/cas4_cond3_4}	
	\caption{}\label{fig:cas4_cond3_4}
\end{figure}
\FloatBarrier
Toutefois, comme nous l'avons expliqué plus avant, le sommet 0 a déjà le nombre d’arêtes maximum (par définition de niveau 5 maximum), dès lors d'une part l’arête 3-4 n'entrave aucune autre construction du graphe, d'autre part, si le graphe peut être résolu avec cette arête, la démonstration sera tout aussi valable sans cette arête. Au chapitre 4, nous prendrons le graphe de la figure~\ref{fig:cas4_cond3_3} comme base de travail, puisque l'arrête 2-4 augmente les contraintes sans limiter les éventuelles autres constructions du graphe.\\
\\
Pas convention, nous appellerons se schéma "le \textbf{cas 4}" dans la suite du document en référence au 3 autres case déjà décrit par \textsc{Kempe}\\
\FloatBarrier

%
% Étendre le graphe
%

\section{Étendre le graphe}
Nous reprenons ici la manière dont les arêtes entre les différents sommets du graphe minimal peuvent être découpées (transformées en chaines) tout en restant dans les conditions du cas~4. Nous les représentons symboliquement par un couple de sommets sur l’arête qui les lie. Par exemple entre le sommet~1~et~6, on peut insérer~2 sommets de couleur~6~et~1. 
\begin{description}
\item [Avant la 1\up{er} permutation].\
La Figure~\ref{fig:cas4_ext1} représente un schéma toujours conforme au cas~4 (l’exception d'\textsc{Heawood}), pour les arêtes directement utilisées pour répondre aux~2 premières conditions du cas~4 (avant la 1ère permutation).
\begin{figure}[!ht]\centering
	\begin{tikzpicture}
	\def \n {5}
	\def \a {90}
	\def \x {360/\n * (\s - 1)+\a}
	\def \radius {1.1cm}
	\def \radium {2.6cm}
	\def \axesize {3.0cm}
	\draw[{\axeline}] (0,{0-\axesize}) -- (0,{\axesize});
	\node[draw,circle,fill=w] (0) at (0,0) {0};
	% les 5 noeud interieur
	\def \s {1} \node[draw,circle,fill=r]	(\s)  at ({\x}:\radius)	{\s};	\path (\s) edge (0);
	\def \s {2} \node[draw,circle,fill=b]	(\s)  at ({\x}:\radius)	{\s};	\path (\s) edge (0);	\path (\s) edge (1);
	\def \s {3} \node[draw,circle,fill=v]	(\s)  at ({\x}:\radius)	{\s};	\path (\s) edge (0);	\path (\s) edge (2);
	\def \s {4} \node[draw,circle,fill=j]	(\s)  at ({\x}:\radius)	{\s};	\path (\s) edge (0);	\path (\s) edge (3);
	\def \s {5} \node[draw,circle,fill=b]	(\s)  at ({\x}:\radius)	{\s};	\path (\s) edge (0);	\path (\s) edge (4);	\path (\s) edge (1);
	% les 3 arc exterieur
	\draw (0,0) ++(168:\radium) arc (168:18:\radium);
	\draw (0,0) ++(18:\radium) arc (18:-90:\radium);
	%arc en 2 morceau...
	\draw (0,0) ++(-90:\radium) arc (-90:-180:\radium);
	\draw (0,0) ++(180:\radium) arc (180:168:\radium);
	% les 3 noeds exterieur	
	\node[draw,circle,fill=j]	(6)   at (160:\radium) 	{6};		
	\node[draw,circle,fill=r]	(7)   at (-90:\radium)   {7};		
	\node[draw,circle,fill=v]	(8)   at (20:\radium) 	{8};		
	\path (1) edge (6);				
	\path (2) edge (6);			
	\path (3) edge (6);
	\path (8) edge (1);
	\path (8) edge (4);			
	\path (8) edge (5);		
	\path (3) edge (7); 			
	\path (4) edge (7);	
	\path (2) edge (6);		
    %complement
	\node[draw,circle,fill=r]	()   at (-25:\radium) 	{};	
	\node[draw,circle,fill=v]	()   at (-45:\radium) 	{};	
	\node[draw,circle,fill=j]	()   at (-135:\radium) 	{};	
	\node[draw,circle,fill=r]	()   at (-155:\radium) 	{};	
	% 3-7-4
	\node[draw,circle,fill=r]   ()   at (0.4,-1.5) 		{};	
	\node[draw,circle,fill=j]	()   at (0.25,-1.9) 	{};	
	\node[draw,circle,fill=r]   ()   at (-0.4,-1.5) 	{};	
    \node[draw,circle,fill=v]	()   at (-0.25,-1.9) 	{};	
	% 1-6-2
	\node[draw,circle,fill=j]   ()   at (-0.7,1) 		{};	
	\node[draw,circle,fill=r]	()   at (-1.2,1.0) 	{};	
	\node[draw,circle,fill=j]   ()   at (-1.55,0.55) 	{};	
	\node[draw,circle,fill=b]	()   at (-1.9,0.7) 	{};	
	% 1-8-5
	\node[draw,circle,fill=v]   ()   at (0.7,1) 		{};	
	\node[draw,circle,fill=r]	()   at (1.2,1.0) 	{};	
	\node[draw,circle,fill=v]   ()   at (1.55,0.55) 	{};	
	\node[draw,circle,fill=b]	()   at (1.9,0.7) 	{};	
	
	%\node[draw] at (\radium,\radium) {permutation 1-3};
\end{tikzpicture}		
	\caption{}\label{fig:cas4_ext1}
\end{figure}
\FloatBarrier

\item [Après la 1\up{ère} permutation arêtes 3-7 et 4-8].\
On peut de la même manière ajouter des sommets sur les arêtes qui sont exploitées après la 1\up{ere} permutation. Dans ce cas, il faut qu’ils aient les couleurs correspondantes aux sommets après permutations. Les~3 arêtes~3-6,~4-8 et~ 6-8 sont concernées. Si l’on souhaite ajouter des sommets sur l’arête~3-6 (resp 4-8), il faut qu’après la $\mathcal{PC}_{53}$ (resp  $\mathcal{PC}_{24}$), la chaine~3-6 (resp 4-8) soit toujours existante. On doit donc, pour l’arête~3-6, insérer~2 sommets de couleur~2 et~3 (et non~3 et 6) et pour l’arête~4-8 insérer~2 sommets de couleur~4 et~5 (et non~4 et 8).
\begin{figure}[!ht]\centering
	\input{./graphe/cas4_ext2}	
	\caption{}\label{fig:cas4_ext2}
\end{figure}
\FloatBarrier

\item [Après la 1\up{ère} permutation arête 6-8].\
Enfin, l’arête 6-8 est particulière selon l’ordre des permutations, elle prendra des couleurs différentes.
Avant permutation elle est vert/jaune, si la 1\up{er} permutation se fait sur la$\overline{\mathcal{C}}_{24}$, elle est bleu/verte, Si la 1\up{ere} permutation se fait sur la $\overline{\mathcal{C}}_{53}$, elle est bleu/jaune. Il est donc impossible de réunir sur une même chaine 3 couleurs, mais il est tout à fait possible d’avoir 2 chaines entre 6 et 8 comme le montre la figure ~\ref{fig:cas4_ext3}.
\begin{figure}[!ht]\centering
	\input{./graphe/cas4_ext3}	
	\caption{}\label{fig:cas4_ext3}
\end{figure}
\FloatBarrier
Les 3 graphes des figures \ref{fig:cas4_ext1}, \ref{fig:cas4_ext2} et \ref{fig:cas4_ext3} sont tous les 3 conformes aux 3 critères décrits pour correspondre au cas 4 (à l’exception d'\textsc{Heawood}).\\
\end{description}


\section{Schéma de Soifer et Fritsch}
Nous reprenons ci dessous une rapide analyse des 2 graphe minimales fréquemment cité lorsque l'on évoque le théorème des 4 couleurs.
\subsection{Schéma de Soifer}
Il est fréquemment représenté sous cette forme.
Seuls 4 sommets (noté A,B,C,D) sont de niveau 5 et pourraient correspondre à notre sommet 0.

\begin{figure}[!ht]\centering
	\begin{tikzpicture}
	%line width=5pt pour les ligne droite
	\def \n {4} 
	\def \a {225}
	\def \x {360/\n * (\s - 1)+\a}
	\def \radius {1.2cm}
	\def \radiusext {2.9cm}
	%\node[draw,circle,fill=w] (0) at (0,0) {0};
	% les 4 noeud exterieur
	\def \s {0} \node[draw,circle,fill=r]	(\s)  at ({\x}:\radiusext)	{A};	
	\def \s {0} \node[draw,circle,fill=r]	(5)  at ({\x}:\radius)  {B};		
	\def \s {1} \node[draw,circle,fill=w]	(\s)  at ({\x}:\radiusext)	{O};	
	\def \s {1} \node[draw,circle,fill=w]	(6)  at ({\x}:\radius)  {O};		
	\def \s {2} \node[draw,circle,fill=w]	(\s)  at ({\x}:\radiusext)	{O};	
	\def \s {2} \node[draw,circle,fill=r]	(7)  at ({\x}:\radius)  {D};		
	\def \s {3} \node[draw,circle,fill=w]	(\s)  at ({\x}:\radiusext)	{O};	
	\def \s {3} \node[draw,circle,fill=w]	(8)  at ({\x}:\radius)  {O};		
						
	\node[draw,circle,fill=r] (9) at (0,-1.4) {C};
	
	\path (0) edge (1);
	\path (0) edge (3);
	\path (0) edge (5);
	\path (0) edge (6);
	\path (0) edge (8);
	\path (1) edge (2);
	\path (2) edge (3);	
	\path (1) edge (6);
	\path (1) edge (9);
	\path (2) edge (7);
	\path (2) edge (9);
	\path (3) edge (7);
	\path (3) edge (8);
	\path (5) edge (6);
	\path (5) edge (7);
	\path (5) edge (8);
	\path (5) edge (9);
	\path (6) edge (9);
	\path (7) edge (9);
	\path (8) edge (7);
	%\node[draw] at (\radium,\radium) {permutation 1-3};
\end{tikzpicture}																																																																		
	\caption{Graph de Soifer\\\footnotesize\textit{représentation habituelle}}\label{fig:soifer}
\end{figure}

Si nous prenons l’hypothèse que A est notre sommet 0, on peut représenter le graphe de multiple manière.
\begin{figure}[!h]\centering
	\begin{changemargin}{-3cm}{-3cm}
		\begin{center}
			\begin{tikzpicture}
	%line width=5pt pour les ligne droite
	\def \n {4} 
	\def \a {225}
	\def \x {360/\n * (\s - 1)+\a}
	\def \radius {1.2cm}
	\def \radiusext {2.9cm}
	%\node[draw,circle,fill=w] (0) at (0,0) {0};
	% les 4 noeud exterieur
	\def \s {0} \node[draw,circle,fill=w]	(0)  at ({\x}:\radiusext)	{0};	
	\def \s {0} \node[draw,circle,fill=r]	(3)  at ({\x}:\radius)  	{1};		
	\def \s {1} \node[draw,circle,fill=j]	(1)  at ({\x}:\radiusext)	{4};	
	\def \s {1} \node[draw,circle,fill=b]	(2)  at ({\x}:\radius)  	{5};		
	\def \s {2} \node[draw,circle,fill=r]	(6)  at ({\x}:\radiusext)	{7};	
	\def \s {2} \node[draw,circle,fill=j]	(7)  at ({\x}:\radius)  	{6};		
	\def \s {3} \node[draw,circle,fill=v]	(5)  at ({\x}:\radiusext)	{3};	
	\def \s {3} \node[draw,circle,fill=b]	(4)  at ({\x}:\radius)  	{2};		
						
	\node[draw,circle,fill=v] (8) at (0,-1.4) {8};
	
	\path (0) edge (1);
	\path (0) edge (2);
	\path (0) edge (3);
	\path (0) edge (4);
	\path (0) edge (5);
	\path (0) edge (5);
	\path (1) edge (2);
	\path (1) edge (6);
	\path (1) edge (8);
	\path (2) edge (3);	
	\path (2) edge (8);
	\path (3) edge (4);
	\path (3) edge (7);
	\path (3) edge (8);
	\path (4) edge (5);
	\path (4) edge (7);
	\path (5) edge (6);
	\path (5) edge (7);
	\path (6) edge (7);
	\path (6) edge (8);
	\path (7) edge (8);
	%\node[draw] at (\radium,\radium) {permutation 1-3};
\end{tikzpicture}																																																																	
			\hspace{15pt}
			\begin{tikzpicture}
	%line width=5pt pour les ligne droite
	\def \co {2.07cm}  
	%coté si rayon=2.9
	\def \cod {1.0cm}
	
	\node[draw,circle,fill=w] (0) at (-\co , \co) {0};
	\node[draw,circle,fill=j] (1) at (-\co ,-\co) {4};
	\node[draw,circle,fill=b] (2) at (-\cod,-\cod) {5};
	\node[draw,circle,fill=r] (3) at ( 0.0 , 0.0) {1};
	\node[draw,circle,fill=b] (4) at ( \cod, \cod) {2};
	\node[draw,circle,fill=v] (5) at ( \co , \co) {3};

	\node[draw,circle,fill=v] (6) at ( 0.5,-1.4) {8};
	\node[draw,circle,fill=r] (7) at ( \co , -\co) {7};
	\node[draw,circle,fill=j] (8) at ( 1.4,-0.5) {6};
	
	\path (0) edge (1);
	\path (0) edge (2);
	\path (0) edge (3);
	\path (0) edge (4);
	\path (0) edge (5);
	\path (1) edge (2);
	\path (1) edge (6);
	\path (1) edge (7);
	\path (2) edge (3);	
	\path (2) edge (6);
	\path (3) edge (4);
	\path (3) edge (6);
	\path (3) edge (8);
	\path (4) edge (5);
	\path (4) edge (8);
	\path (5) edge (7);
	\path (5) edge (8);
	\path (6) edge (7);
	\path (6) edge (8);
	\path (7) edge (8);
	%\node[draw] at (\radium,\radium) {permutation 1-3};
\end{tikzpicture}																																																																	
			\hspace{15pt}
			\input{./graphe/soifer_3}
		\end{center}
	\end{changemargin}
\end{figure}
\begin{figure}[!h]\centering
	\begin{changemargin}{-3cm}{-3cm}
		\begin{center}
			\begin{tikzpicture}
	%line width=5pt pour les ligne droite
	\def \co {2.07cm}  
	%coté si rayon=2.9
	\def \cod {1.0cm}
	
	\node[draw,circle,fill=w] (0) at (0 , 0) {0};
	\node[draw,circle,fill=r] (1) at ( \cod ,-\cod) {1};
	\node[draw,circle,fill=b] (2) at ( \cod, \cod)	{2};
	\node[draw,circle,fill=v] (3) at ( 0 , \co) 	{3};
	\node[draw,circle,fill=j] (4) at (-\co ,0) 		{4};
	\node[draw,circle,fill=b] (5) at (-\cod,-\cod) 	{5};

	\node[draw,circle,fill=j] (6) at ( \co,0) 		{6};
	\node[draw,circle,fill=r] (7) at ( -\co , \co) 	{7};
	\node[draw,circle,fill=v] (8) at ( 0,-\co) 		{8};
	
	\path (0) edge (1);
	\path (0) edge (2);
	\path (0) edge (3);
	\path (0) edge (4);
	\path (0) edge (5);
	\path (1) edge (2);
	\path (1) edge (6);
	\path (1) edge (5);
	\path (1) edge (8);
	\path (2) edge (3);	
	\path (2) edge (6);
	%\path (3) edge (4);
	\path (3) edge (7);
	\path (4) edge (5);
	\path (4) edge (7);
	\path (4) edge[looseness=2, bend left=-20] (8);
	\path (5) edge (8);
	\path (3) edge[looseness=2, bend left=20] (3);
	%\path (6) edge (7);
	\path (6) edge[looseness=2, bend left=20] (8);
	%\path (6) edge (8);
	\path (3) edge[looseness=2, bend left=20] (6);
	%\path (7) edge (8);
	\path (6) edge[looseness=2, bend left=-50] (7);
	\path (7) edge[looseness=2, bend left=-50] (8);

	%\node[draw] at (\radium,\radium) {permutation 1-3};
\end{tikzpicture}	
			\hspace{15pt}
			\begin{tikzpicture}
	\def \n {5}
	\def \a {180}
	\def \x {360/\n * (\s - 1)+\a}
	\def \radius {1.1cm}
	\def \radium {2.2cm}
	\node[draw,circle,fill=w] (0) at (0,0) {0};
	% les 5 noeud interieur
	\def \s {1} \node[draw,circle,fill=j]	(\s)  at ({\x}:\radius)	{4};	\path (\s) edge (0);
	\def \s {2} \node[draw,circle,fill=b]	(\s)  at ({\x}:\radius)	{5};	\path (\s) edge (0);	\path (\s) edge (1);
	\def \s {3} \node[draw,circle,fill=r]	(\s)  at ({\x}:\radius)	{1};	\path (\s) edge (0);	\path (\s) edge (2);
	\def \s {4} \node[draw,circle,fill=b]	(\s)  at ({\x}:\radius)	{2};	\path (\s) edge (0);	\path (\s) edge (3);
	\def \s {5} \node[draw,circle,fill=v]	(\s)  at ({\x}:\radius)	{3};	\path (\s) edge (0);	%\path (\s) edge (4);	\path (\s) edge (1);
	% les 3 arc exterieur
	\draw (0,0) ++(168:\radium) arc (168:18:\radium);
	\draw (0,0) ++(18:\radium) arc (18:-90:\radium);
	%arc en 2 morceau...
	\draw (0,0) ++(-90:\radium) arc (-90:-180:\radium);
	\draw (0,0) ++(180:\radium) arc (180:168:\radium);
	% les 3 noeds exterieur	
	\node[draw,circle,fill=v]	(6)   at (70+\a:\radium) 	{8};		
	\node[draw,circle,fill=r]	(7)   at (-35+\a:\radium)   {7};		
	\node[draw,circle,fill=j]	(8)   at (-150+\a:\radium) 	{6};		
	\path (1) edge (6);				
	\path (1) edge (7);			
	\path (2) edge (6);
	\path (3) edge (6);
	\path (3) edge (8);			
	\path (4) edge (8);		
	\path (5) edge (7); 			
	\path (5) edge (8);		

	%\node[draw] at (\radium,\radium) {permutation 1-3};
\end{tikzpicture}	
			\hspace{15pt}
			\input{./graphe/soifer_6}
		\end{center}
	\end{changemargin}
	\begin{changemargin}{1.75cm}{1.75cm}
		\caption{Soifer - équivalant}\label{fig:Soifer_anamorphose}
	\end{changemargin}
\end{figure}
\FloatBarrier

On peut constater que nous sommes bien dans le cas où la méthode de \textsc{Kempe} ne fonctionne pas (Exception d'\textsc{Heawood}).Après la $\mathcal{PC}_{24}$ puis la $\mathcal{PC}_{53}$ il reste 4 couleurs autour de 0.
La $\mathcal{PC}_{53}$ puis la $\mathcal{PC}_{24}$ donne également 4 couleurs autour de 0.\\
Nous pouvons également constater que ce graphe correspond au graphe construit pour répondre au 3 conditions du cas 4 (la  Figure~\ref{fig:cas4_cond3_4} est la même que le figure Figure~\ref{fig:Soifer_anamorphose}). 
\\
\subsection{Schéma de Fritsch}
\begin{figure}[!h]\centering
	\begin{changemargin}{-3cm}{-3cm}
		\begin{center}
			\begin{tikzpicture}
	\def \n {5}
	\def \a {90}
	\def \x {360/\n * (\s - 1)+\a}
	\def \radius {1.1cm}
	\def \radium {2.5cm}
	\node[draw,circle,fill=r] (1) at (0,0) {};
	% les 5 noeud interieur
	\def \s {2} \node[draw,circle,fill=r]	(2)  at ({\x}:\radius)	{};	\path (2) edge (1);
	\def \s {3} \node[draw,circle,fill=r]	(0)  at ({\x}:\radius)	{};	\path (0) edge (1);	
	\def \s {4} \node[draw,circle,fill=r]	(5)  at ({\x}:\radius)	{};	\path (5) edge (1);	
	\def \s {5} \node[draw,circle,fill=r]	(7)  at ({\x}:\radius)	{};	\path (7) edge (1);	
	\node[draw,circle,fill=r]	(8)   at (0,{\radium}) 	{};	
	\node[draw,circle,fill=r]	(4)   at (0,-2) 	{};	
	\node[draw,circle,fill=r]	(3)   at (-2,{0-\radium}) 	{};	
	\node[draw,circle,fill=r]	(6)   at (2,{0-\radium}) 	{};	
	\path (0) edge (2);				
	\path (0) edge (3);	
	\path (0) edge (4);				
	\path (0) edge (5);				
	\path (1) edge (8);				
	\path (2) edge (3);			
	\path (2) edge (8);
	\path (3) edge (4);			
	\path (3) edge (6); 			
	\path (4) edge (5);		
	\path (4) edge (6);		
	\path (5) edge (6);		
	\path (5) edge (7);				
	\path (6) edge (7);		
	\path (7) edge (8);
	\path (3) edge[looseness=1, bend left=25] (8);
	\path (6) edge[looseness=1, bend left=-25] (8);

	%\node[draw] at (\radium,\radium) {permutation 1-3};
\end{tikzpicture}	
			\hspace{15pt}
			\input{./graphe/fritsch_2}
			\hspace{15pt}
			\begin{tikzpicture}
	\def \n {5}
	\def \a {-150}
	\def \x {360/\n * (\s - 1)+\a}
	\def \radius {1.1cm}
	\def \radium {2.2cm}

	\draw[{\axeline}] (-1.4,-0.75) -- (3.5,1.8);
	
	\node[draw,circle,fill=w] (0) at (0,0) {0};
	% les 5 noeud interieur
	\def \s {1} \node[draw,circle,fill=r]	(\s)  at ({\x}:\radius)	{\s};	\path (\s) edge (0);
	\def \s {2} \node[draw,circle,fill=b]	(\s)  at ({\x}:\radius)	{\s};	\path (\s) edge (0);	\path (\s) edge (1);
	\def \s {3} \node[draw,circle,fill=v]	(\s)  at ({\x}:\radius)	{\s};	\path (\s) edge (0);	\path (\s) edge (2);
	\def \s {4} \node[draw,circle,fill=j]	(\s)  at ({\x}:\radius)	{\s};	\path (\s) edge (0);	\path (\s) edge (3);
	\def \s {5} \node[draw,circle,fill=b]	(\s)  at ({\x}:\radius)	{\s};	\path (\s) edge (0);	\path (\s) edge (4);	\path (\s) edge (1);
	% les 3 neud exterieur
	\node[draw,circle,fill=j]	(6)   at (2,{0-\radium})	{6};	
	\node[draw,circle,fill=r]	(7)   at (2,1) 				{7};	
	\node[draw,circle,fill=v]	(8)   at (0,{\radium})		{8};	
	%
	\path (3) edge (6);
	\path (3) edge (7);			
	\path (2) edge (6);			
	\path (4) edge (7);	
	\path (4) edge (8);	
	\path (5) edge (8);		
	\path (6) edge (7);		
	\path (7) edge (8);		
	%
	\path (1) edge[looseness=1, bend left=-40] (6);
	\path (1) edge[looseness=1, bend left=40]  (8);
	\path (6) edge[looseness=1.5, bend left=-80] (8);
	%\node[draw] at (\radium,\radium) {permutation 1-3};
\end{tikzpicture}	
		\end{center}
	\end{changemargin}
\end{figure}
\FloatBarrier	

\begin{figure}[!h]\centering
	\begin{changemargin}{-3cm}{-3cm}
		\begin{center}
			\begin{tikzpicture}
\def \n {5}
\def \a {-150}
\def \x {360/\n * (\s - 1)+\a}
\def \radius {1.1cm}
\def \radium {2.2cm}
\draw[{\axeline}] (-2.5,-1.4) -- (2.6,1.5);
\node[draw,circle,fill=w] (0) at (0,0) {0};
\draw (0,0) circle ({\radium});
% les 5 noeud interieur
\def \s {1} \node[draw,circle,fill=r]	(\s)  at ({\x}:\radius)	{\s};	\path (\s) edge (0);
\def \s {2} \node[draw,circle,fill=b]	(\s)  at ({\x}:\radius)	{\s};	\path (\s) edge (0);	\path (\s) edge (1);
\def \s {3} \node[draw,circle,fill=v]	(\s)  at ({\x}:\radius)	{\s};	\path (\s) edge (0);	\path (\s) edge (2);
\def \s {4} \node[draw,circle,fill=j]	(\s)  at ({\x}:\radius)	{\s};	\path (\s) edge (0);	\path (\s) edge (3);
\def \s {5} \node[draw,circle,fill=b]	(\s)  at ({\x}:\radius)	{\s};	\path (\s) edge (0);	\path (\s) edge (4);	\path (\s) edge (1);
% les 3 neud exterieur
\node[draw,circle,fill=j]	(6)   at (-82:{\radium}) 	{6};	
\node[draw,circle,fill=r]	(7)   at (30:{\radium}) 	{7};	
\node[draw,circle,fill=v]	(8)   at (145:{\radium})	{8};	
%
\path (1) edge (6);
\path (1) edge (8);			
\path (2) edge (6);
\path (3) edge (6);
\path (3) edge (7);			
\path (4) edge (7);	
\path (4) edge (8);		
\path (5) edge (8);		
		
%\node[draw] at (\radium,\radium) {permutation 1-3};
\end{tikzpicture}	
			\hspace{15pt}
			\begin{tikzpicture}
	\def \n {5}
	\def \a {90}
	\def \x {360/\n * (\s - 1)+\a}
	\def \radius {1.1cm}
	\def \radium {2.2cm}
	\def \axesize {2.7cm}
	\draw[{\axeline}] (0,{0-\axesize}) -- (0,{\axesize});
	\node[draw,circle,fill=w] (0) at (0,0) {0};
	% les 5 noeud interieur
	\def \s {1} \node[draw,circle,fill=r]	(\s)  at ({\x}:\radius)	{\s};	\path (\s) edge (0);
	\def \s {2} \node[draw,circle,fill=b]	(\s)  at ({\x}:\radius)	{\s};	\path (\s) edge (0);	\path (\s) edge (1);
	\def \s {3} \node[draw,circle,fill=v]	(\s)  at ({\x}:\radius)	{\s};	\path (\s) edge (0);	\path (\s) edge (2);
	\def \s {4} \node[draw,circle,fill=j]	(\s)  at ({\x}:\radius)	{\s};	\path (\s) edge (0);	\path (\s) edge (3);
	\def \s {5} \node[draw,circle,fill=b]	(\s)  at ({\x}:\radius)	{\s};	\path (\s) edge (0);	\path (\s) edge (4);	\path (\s) edge (1);
	% les 3 arc exterieur
	\draw (0,0) ++(168:\radium) arc (168:18:\radium);
	\draw (0,0) ++(18:\radium) arc (18:-90:\radium);
	%arc en 2 morceau...
	\draw (0,0) ++(-90:\radium) arc (-90:-180:\radium);
	\draw (0,0) ++(180:\radium) arc (180:168:\radium);
	% les 3 noeds exterieur	
	\node[draw,circle,fill=j]	(6)   at (160:\radium) 	{6};		
	\node[draw,circle,fill=r]	(7)   at (-90:\radium)   {7};		
	\node[draw,circle,fill=v]	(8)   at (20:\radium) 	{8};		
	\path (1) edge (6);				
	\path (2) edge (6);			
	\path (3) edge (6);
	\path (8) edge (1);
	\path (8) edge (4);			
	\path (8) edge (5);		
	\path (3) edge (7); 			
	\path (4) edge (7);		

	%\node[draw] at (\radium,\radium) {permutation 1-3};
\end{tikzpicture}	
			\caption{Fritsch - équivalant}
			\label{fig:Fritsch_anamorphose}
		\end{center}
	\end{changemargin}
\end{figure}
\FloatBarrier	
Nous pouvons constater que ce graphe correspond au graphe construit pour répondre aux 3 conditions du cas 4 (la Figure~\ref{fig:cas4_cond3_3} est la même que la figure~\ref{fig:Fritsch_anamorphose}). 
Comme expliqué plus haut, l’arête 3-4 n'est pas indispensable.
La différence entre le graphe de Fritch et celui de Soifer se limite à cette seule arête reliant les sommets 3 et 4 (inexistante chez Soifer).\\




%*******************************************************%*******************************************************%*******************************************************%*******************************************************%*******************************************************

\section{Résoudre le "cas 4"- demonstration}
Il ne nous reste plus qu'à résoudre le cas 4  (l'exception d'\textsc{Heawood}). Si l'on peut montrer qu'il y a toujours moyen de sortir du cas 4 (et dès lors de pouvoir limiter à 3 les couleurs des voisins du sommet 0, par l'une des 3 méthodes proposée par \textsc{Kempe}), alors le théorème des 4 couleurs est démontré.
\\
\begin{description}
	\item [Nous savons que :]
La proposition de Kempe, bien que comportant une erreur, expose une méthodologie et un traitement pertinent pour résoudre le théorème des 4 couleurs.
Il n'existe qu'une exception, un seul cas non correctement traité dans la démonstration de Kempe. Nous l'avons appelé « le cas 4 ». Il a été défini avec précision au point précédant « l'exeception de Heawood ».\\

	\item [Nous souhaitons démontrer :]
qu’il y a toujours moyen de sortir du cas 4 et donc de pouvoir limiter à 3 les couleurs des voisins du sommet 0, par l’une des 3 méthodes proposées par Kempe.\\


	\item [Le principe de la démonstration :]
Nous allons montrer que, a l'aide de permutations de chaines, on peut réaliser une chaine commune entre le sommet 5 et le sommet 3, qui empêche le sommet 4 d'avoir une chaine commune avec le sommet 2. L’inexistence de cette chaine a pour conséquence de toujours sortir du cas 4! De manière imagée, Nous allons encercler le sommet 4 par une barrière qui relie le sommet 5 et le sommet 3 (Figure~\ref{fig:cas4_r15_4}).
\end{description}
\FloatBarrier
\begin{figure}[!ht]\centering
	\begin{tikzpicture}[scale=0.85, every node/.style={scale=0.85}]
	\def \n {5}
	\def \a {90}
	\def \x {360/\n * (\s - 1)+\a}
	\def \radius {1.1cm}
	\def \radium {2.6cm}
	\def \axesize {3.0cm}
	\draw[{\axeline}] (0,{0-\axesize}) -- (0,{\axesize});
	\node[draw,circle,fill=w] (0) at (0,0) {0};
	% les 5 noeud interieur
	\def \s {1} \node[draw,circle,fill=b]	(\s)  at ({\x}:\radius)	{\s};	\path (\s) edge (0);
	\def \s {2} \node[draw,circle,fill=v]	(\s)  at ({\x}:\radius)	{\s};	\path (\s) edge (0);	\path (\s) edge (1);
	\def \s {3} \node[draw,circle,fill=r]	(\s)  at ({\x}:\radius)	{\s};	\path (\s) edge (0);	\path (\s) edge (2);
	\def \s {4} \node[draw,circle,fill=j]	(\s)  at ({\x}:\radius)	{\s};	\path (\s) edge (0);	\path (\s) edge (3);
	\def \s {5} \node[draw,circle,fill=r]	(\s)  at ({\x}:\radius)	{\s};	\path (\s) edge (0);	\path (\s) edge (4);	
	\path (\s) edge (1);
	% les 3 arc exterieur
	\draw (0,0) ++(168:\radium) arc (168:18:\radium);
	\draw (0,0) ++(18:\radium) arc (18:-90:\radium);
	%arc en 2 morceau...
	\draw (0,0) ++(-90:\radium) arc (-90:-180:\radium);
	\draw (0,0) ++(180:\radium) arc (180:168:\radium);
	% les 3 noeds exterieur	
	\node[draw,circle,fill=j]	(6)   at (160:\radium) 	{6};		
	\node[draw,circle,fill=b]	(7)   at (-90:\radium)   {7};		
	\node[draw,circle,fill=v]	(8)   at (20:\radium) 	{8};		
	\path (1) edge (6);				
	\path (1) edge (8);
	\path (2) edge (6);			
	\path (3) edge (6);
	\path (3) edge[bicolor={b}{r},line width=4pt] (7); 			
	\path (4) edge (7);	
	%\path (4) edge (8);
	\path (4) edge[looseness=1.5,bend right=40] (8);	
	\path (5) edge (8);		
    %complement
	% 4-8
	\node[draw,circle,fill=r,scale=0.6]  (9)   at (1.45,-0.8)	{9};	
 	\node[draw,circle,fill=j,scale=0.5]	(10)   at (2.3,-0.2) 	{10};	
 	\node[draw,circle,fill=b,scale=0.5]	(11)   at (1.9,0.1) 	{11};
	\path (7) edge[bicolor={b}{r},line width=4pt] (9);	
	\path (11) edge[bicolor={b}{r},line width=4pt] (9);
	\path (5) edge[bicolor={b}{r},line width=4pt] (11);

\end{tikzpicture}		
	\caption{le sommet 4 n'a plus de $\overline{\mathcal{C}}_{42}$ nous ne somme donc plus dans le cas 4}\label{fig:cas4_r15_4}
\end{figure}
\FloatBarrier

%  ******************************************************
\subsection{demonstration express}
%  ******************************************************
\begin{description}
\item [Démonstration :] Nous allons utiliser une méthode simple en 2 temps (et quelques "travaux" intermédiaires).
La 1ère étape (Figure~\ref{fig:cas4_etap15_3}) va valider l’existence d'une chaine entre le sommet 5 et le sommet 7 (intersection entre $\overline{\mathcal{C}}_{13}$ et $\overline{\mathcal{C}}_{14}$), la 2ème étape (Figure~\ref{fig:cas4_r15_4}) va prolonger cette chaine jusqu'au sommet 3.
\end{description}
\FloatBarrier
\begin{figure}[!ht]\centering
	\begin{changemargin}{-3cm}{-5cm}
	\begin{center}
		\begin{subfigure}{0.4\textwidth}
			\input{./graphe_r/cas4_r15_2}
			\caption{}\label{fig:cas4_etap15_2}
		\end{subfigure}
		\begin{subfigure}{0.4\textwidth}
			%\hspace{105pt}
			\input{./graphe_r/cas4_r15_3}
			\caption{}\label{fig:cas4_etap15_3}
		\end{subfigure}
		\begin{subfigure}{0.4\textwidth}
			%\hspace{15pt}
			\begin{tikzpicture}[scale=0.85, every node/.style={scale=0.85}]
	\def \n {5}
	\def \a {90}
	\def \x {360/\n * (\s - 1)+\a}
	\def \radius {1.1cm}
	\def \radium {2.6cm}
	\def \axesize {3.0cm}
	\draw[{\axeline}] (0,{0-\axesize}) -- (0,{\axesize});
	\node[draw,circle,fill=w] (0) at (0,0) {0};
	% les 5 noeud interieur
	\def \s {1} \node[draw,circle,fill=b]	(\s)  at ({\x}:\radius)	{\s};	\path (\s) edge (0);
	\def \s {2} \node[draw,circle,fill=v]	(\s)  at ({\x}:\radius)	{\s};	\path (\s) edge (0);	\path (\s) edge (1);
	\def \s {3} \node[draw,circle,fill=r]	(\s)  at ({\x}:\radius)	{\s};	\path (\s) edge (0);	\path (\s) edge (2);
	\def \s {4} \node[draw,circle,fill=j]	(\s)  at ({\x}:\radius)	{\s};	\path (\s) edge (0);	\path (\s) edge (3);
	\def \s {5} \node[draw,circle,fill=r]	(\s)  at ({\x}:\radius)	{\s};	\path (\s) edge (0);	\path (\s) edge (4);	
	\path (\s) edge (1);
	% les 3 arc exterieur
	\draw (0,0) ++(168:\radium) arc (168:18:\radium);
	\draw (0,0) ++(18:\radium) arc (18:-90:\radium);
	%arc en 2 morceau...
	\draw (0,0) ++(-90:\radium) arc (-90:-180:\radium);
	\draw (0,0) ++(180:\radium) arc (180:168:\radium);
	% les 3 noeds exterieur	
	\node[draw,circle,fill=j]	(6)   at (160:\radium) 	{6};		
	\node[draw,circle,fill=b]	(7)   at (-90:\radium)   {7};		
	\node[draw,circle,fill=v]	(8)   at (20:\radium) 	{8};		
	\path (1) edge (6);				
	\path (1) edge (8);
	\path (2) edge (6);			
	\path (3) edge (6);
	\path (3) edge[bicolor={b}{r},line width=4pt] (7); 			
	\path (4) edge (7);	
	%\path (4) edge (8);
	\path (4) edge[looseness=1.5,bend right=40] (8);	
	\path (5) edge (8);		
    %complement
	% 4-8
	\node[draw,circle,fill=r,scale=0.6]  (9)   at (1.45,-0.8)	{9};	
 	\node[draw,circle,fill=j,scale=0.5]	(10)   at (2.3,-0.2) 	{10};	
 	\node[draw,circle,fill=b,scale=0.5]	(11)   at (1.9,0.1) 	{11};
	\path (7) edge[bicolor={b}{r},line width=4pt] (9);	
	\path (11) edge[bicolor={b}{r},line width=4pt] (9);
	\path (5) edge[bicolor={b}{r},line width=4pt] (11);

\end{tikzpicture}	
			\caption{}\label{fig:cas4_etap15_4}
		\end{subfigure}
	\end{center}
	\caption{permutation de $\mathcal{PC}_{12}$ (b) puis permutation de $\mathcal{PC}_{23}$ (c)}
	\end{changemargin}
\end{figure}	
\FloatBarrier
\begin{description}
\item [La 1ere étape] consiste à permuter la chaine des couleurs des sommets 1 et 2 ($\mathcal{PC}_{12}$).\
Nous avons pu constater que la ou les intersections entre les chaines $\overline{\mathcal{C}}_{13}$ et $\overline{\mathcal{C}}_{14}$ sont indispensables pour être dans le cas de figure du cas 4. Notre première étape va soit les supprimer, soit nous garantir qu'il existe une chaine commune entre le sommet 5 et une intersection des $\overline{\mathcal{C}}_{13}$ et $\overline{\mathcal{C}}_{14}$ (Figure~\ref{fig:cas4_etap15_3}).
	\begin{changemargin}{0cm}{0cm}

\textbf{Après cette permutation,}
\begin{description}
\item[Soit :] nous ne sommes plus dans le cas 4 (la démonstration est faite pour ce graphe).
\item[Soit :] il existe au moins un sommet, que nous appelons x, ayant les conditions suivantes :\\
\item[]il est à l’intersection de la $\mathcal{C}_{13}$ et la $\mathcal{C}_{14}$.
Ce constat découle de notre première permutation. S’il n’existe plus d’intersection entre la $\mathcal{C}_{13}$ et la $\mathcal{C}_{14}$, nous ne sommes plus dans le cas~4 (et la démonstration est faite pour se graphe).\\
\item[]La $\overline{\mathcal{C}}_{x3}$ n’a pas d’autre intersection avec la $\mathcal{C}_{14}$ que le sommet «~x~».\\
Si un sommet y sur la $\overline{\mathcal{C}}_{x3}$ fait également partie des $\overline{\mathcal{C}}_{13}$ et $\overline{\mathcal{C}}_{14}$, alors il existe une $\overline{\mathcal{C}}_{y3}$ contenue dans $\overline{\mathcal{C}}_{x3}$ tel que $\overline{\mathcal{C}}_{y3}$ n'a pas d’autre intersection avec la $\overline{\mathcal{C}}_{14}$ que le sommet~«~y~».\\
\end{description}
	\end{changemargin}

\textit{La moitié du travail est ainsi fait, nous avons une $\overline{\mathcal{C}}_{x5}$. Il ne nous reste plus qu'à prolonger cette chaine jusqu'au sommet 3 pour "encercler" le sommet 4 et sortir du cas 4.}\\

\item [La 2ème étape] consiste à permuter la $\overline{\mathcal{C}}_{23}$ ; ce qui nous amène à la
configuration suivante (Figure~\ref{fig:cas4_etap15_4}):\

C’est le sommet 4 qui a maintenant 2 voisins de couleur identique. C’est ce même sommet 4
qui doit avoir 2 chaines communes avec les sommets 1 et 2 pour satisfaire la 1ere condition du cas 4. 
Dans un cas simple, notre objectif est atteint  (Figure~\ref{fig:cas4_etap15_3}), la permutation crée une $\overline{\mathcal{C}}_{35}$ qui sépare et isole le sommet 4 du sommet 2. Nous sommes bien sorti du cas 4, la $\overline{\mathcal{C}}_{42}$ n’existe pas et la $\mathcal{PC}_{42}$ permet de ramener les couleurs autour du sommet 0 a 3  (la démonstration est finie – Figure~\ref{fig:cas4_etap15_4}).\\

%
%
%
%
Si nous nous intéressons plus en détail a la $\overline{\mathcal{C}}_{x3}$, elle peut comme dans l'exemple de la figure \ref{fig:cas4_etap15_4} ne comporter que 2 sommets (x~et~3), mais elle peut également être constituée de multiples sommets ou comporter plusieurs $\overline{\mathcal{C}}_{x3}$ distinctes.\\

Revenons à notre schéma après la 1ère permutation (${\mathcal{PC}}_{12}$). La $\overline{\mathcal{C}}_{x3}$ est de couleur des sommet « x » et 3. Par définition, on sait que la $\overline{\mathcal{C}}_{x3}$ n’a pas d’élément commun avec la $\overline{\mathcal{C}}_{14}$ autre que x. La $\overline{\mathcal{C}}_{x3}$ est donc confinée à l’intérieur de la $\overline{\mathcal{C}}_{14}$ (figure 32).

On peut donc être certain que toutes permutations de chaines comportant un des sommet de la $\overline{\mathcal{C}}_{x3}$ et de la couleur des sommets 2 et 3 ne pourras pas affecter les sommet 1,4 et 5 (voisin de 0). Les sommets 1 et 4 parce qu’il sont d'une des 2 autre couleur (que les sommet 2 et 3), le somme 5 parce qu’il est séparer de ces sommet par la $\overline{\mathcal{C}}_{14}$.\\

Pour garantir lors de la ${\mathcal{PC}}_{23}$ que la $\overline{\mathcal{C}}_{35}$ existera bien, nous allons parcourir les sommets de la couleur du sommet 3 sur la $\overline{\mathcal{C}}_{x3}$ en partant du sommet x. Nous appelons se sommet y. S'il existe plusieurs $\overline{\mathcal{C}}_{x3}$, nous n'en parcourrons qu'une seule. 
Pour chaque sommet y, de la $\overline{\mathcal{C}}_{x3}$ initiale, 3 cas de figure peuvent se présenter :\\
- la $\overline{\mathcal{C}}_{y2}$ existe : Nous ne faisons rien. Le sommet y, puisqu'il a une $\overline{\mathcal{C}}_{y2}$, changera de couleur lors de la $\overline{\mathcal{C}}_{23}$.\\
- La $\overline{\mathcal{C}}_{y2}$ n'existe pas. Nous faisons une ${\mathcal{PC}}_{y2}$. Ce sommet ne sera pas affecté par la future $\overline{\mathcal{C}}_{23}$, il prend par ailleurs la couleur du sommet 2 (ou 5) qui permet in fine de créer la $\overline{\mathcal{C}}_{35}$.\\
- Il a déjà changé de couleur lors de la permutation d'un sommet précédant. On ne fait rien ! (il a déjà la couleur du sommet 2 (ou 5)).\\


Après ce traitement, la ${\mathcal{PC}}_{23}$ nous amène toujours dans un graphe ayant ces 2 particularités :\

C’est le sommet 4 qui a maintenant 2 voisins de couleur identique. C’est ce même sommet 4
qui doit avoir 2 chaines communes avec les sommets 1 et 2 pour satisfaire la 1ère condition du cas 4, or la $\overline{\mathcal{C}}_{42}$ ne peut pas exister puisqu’il existe une $\overline{\mathcal{C}}_{35}$.\\

Conclusion : Nous ne somme plus dans le cas 4, la ${\mathcal{PC}}_{42}$ permet de ramener le nombre de couleurs des sommets autour du sommet 0 a 3. 
La démonstration est dés lors complète !
\end{description}












%  ******************************************************
\subsection{la même demonstration avec exemples et ilustration}
%  ******************************************************
\begin{description}
%  ******************************************************
% 1er etapes
%  ******************************************************
\item[La 1\up{ere} étape] consiste à permuter la chaine des couleurs des sommets 1 et 2 ($\mathcal{PC}_{12}$).\
Cela a pour conséquence de casser la $\overline{\mathcal{C}}_{13}$ et la $\overline{\mathcal{C}}_{14}$ là où les intersections entre ces 2 chaines n'ont pas de chaines communes avec le sommet 2 (ou 5).\\

Bien que à priori, les sommets 1, 2 et 5 soient isolés du sommet 7 (figure \ref{fig:cas4_etap1_1}) par une $\overline{\mathcal{C}}_{12}$, une chaine peut être réalisée soit en traversant la$\overline{\mathcal{C}}_{48}$ (ou en symétrie la $\overline{\mathcal{C}}_{36}$) (figure \ref{fig:cas4_etap1_2}) soit en traversant la $\overline{\mathcal{C}}_{68}$ (figure \ref{fig:cas4_etap1_3})  tout en restant dans le cas 4 (voir chapitre précédant - Étendre le graphe).
\FloatBarrier
\begin{figure}[!ht]\centering
	\begin{changemargin}{-5cm}{-3cm}
	\begin{center}
		\begin{subfigure}{0.4\textwidth}
			\begin{tikzpicture}
	%line width=5pt pour les ligne droite
	\def \n {5}
	\def \a {90}
	\def \x {360/\n * (\s - 1)+\a}
	\def \radius {1.1cm}
	\def \radium {2.2cm}
	\node[draw,circle,fill=w] (0) at (0,0) {0};
	% les 5 noeud interieur
	\def \s {1} \node[draw,circle,fill=r]	(\s)  at ({\x}:\radius)	{\s};	\path (\s) edge (0);
	\def \s {2} \node[draw,circle,fill=b]	(\s)  at ({\x}:\radius)	{\s};	\path (\s) edge (0);	\path (\s) edge (1);
	\def \s {3} \node[draw,circle,fill=v]	(\s)  at ({\x}:\radius)	{\s};	\path (\s) edge (0);	\path (\s) edge (2);
	\def \s {4} \node[draw,circle,fill=j]	(\s)  at ({\x}:\radius)	{\s};	\path (\s) edge (0);	\path (\s) edge[line width=3pt] (3);
	\def \s {5} \node[draw,circle,fill=b]	(\s)  at ({\x}:\radius)	{\s};	\path (\s) edge (0);	\path (\s) edge (4);	\path (\s) edge (1);
	% les 3 arc exterieur
	\draw[line width=3pt] (0,0) ++(168:\radium) arc (168:18:\radium);
	\draw (0,0) ++(18:\radium) arc (18:-90:\radium);
	%arc en 2 morceau...
	\draw (0,0) ++(-90:\radium) arc (-90:-180:\radium);
	\draw (0,0) ++(180:\radium) arc (180:168:\radium);
	% les 3 noeds exterieur	
	\def \t {1} \node[draw,circle,fill=j]	(6)   at (162:\radium) {6};		
	\def \t {2} \node[draw,circle,fill=r]	(7)   at (270:\radium) 	{7};		
	\def \t {3} \node[draw,circle,fill=v]	(8)   at (18:\radium) 	{8};		
	\path (1) edge (6);				
	\path (1) edge (8);			
	\path (2) edge (6);		
	\path (3) edge[line width=3pt] (6);
	\path (3) edge (7);			
	\path (4) edge (7); 			
	\path (4) edge[line width=3pt] (8);		
	\path (5) edge (8);		
	%\node[draw] at (\radium,\radium) {permutation 1-3};
\end{tikzpicture}
			\caption{}\label{fig:cas4_etap1_1}
		\end{subfigure}
		\begin{subfigure}{0.3\textwidth}
			%\hspace{105pt}
			\input{./graphe_r/cas4_r5_1}
			\caption{}\label{fig:cas4_etap1_2}
		\end{subfigure}
		\begin{subfigure}{0.1\textwidth}
			%\hspace{15pt}
			\begin{tikzpicture}[scale=0.85, every node/.style={scale=0.85}]
	\def \n {5}
	\def \a {90}
	\def \x {360/\n * (\s - 1)+\a}
	\def \radius {1.1cm}
	\def \radium {2.6cm}
	\def \axesize {3.0cm}
	\draw[{\axeline}] (0,{0-\axesize}) -- (0,{\axesize});
	\node[draw,circle,fill=w] (0) at (0,0) {0};
	% les 5 noeud interieur
	\def \s {1} \node[draw,circle,fill=r]	(\s)  at ({\x}:\radius)	{\s};	\path (\s) edge (0);
	\def \s {2} \node[draw,circle,fill=b]	(\s)  at ({\x}:\radius)	{\s};	\path (\s) edge (0);	\path (\s) edge[bicolor={b}{r},line width=4pt] (1);
	\def \s {3} \node[draw,circle,fill=v]	(\s)  at ({\x}:\radius)	{\s};	\path (\s) edge (0);	\path (\s) edge (2);
	\def \s {4} \node[draw,circle,fill=j]	(\s)  at ({\x}:\radius)	{\s};	\path (\s) edge (0);	\path (\s) edge (3);
	\def \s {5} \node[draw,circle,fill=b]	(\s)  at ({\x}:\radius)	{\s};	\path (\s) edge (0);	\path (\s) edge (4);	\path (\s) edge[bicolor={b}{r},line width=4pt] (1);
	% les 3 arc exterieur
	%\draw (0,0) ++(168:\radium) arc (168:18:\radium);
	\draw (0,0) ++(18:\radium) arc (18:-90:\radium);
	%arc en 2 morceau...
	\draw (0,0) ++(-90:\radium) arc (-90:-180:\radium);
	\draw (0,0) ++(180:\radium) arc (180:168:\radium);
	% les 3 noeds exterieur	
	\node[draw,circle,fill=j]	(6)   at (160:\radium) 	{6};		
	\node[draw,circle,fill=r]	(7)   at (-90:\radium)   {7};		
	\node[draw,circle,fill=v]	(8)   at (20:\radium) 	{8};		
	\path (1) edge (6);				
	\path (2) edge (6);			
	\path (3) edge (6);
	\path (8) edge (1);
	\path (8) edge (4);			
	\path (8) edge (5);		
	\path (3) edge (7); 			
	\path (4) edge (7);	
	\path (2) edge (6);		
%    %complement
%	\node[draw,circle,fill=r]	()   at (-25:\radium) 	{};	
%	\node[draw,circle,fill=v]	()   at (-45:\radium) 	{};	
%	\node[draw,circle,fill=j]	()   at (-135:\radium) 	{};	
%	\node[draw,circle,fill=r]	()   at (-155:\radium) 	{};	
%	% 3-7-4
%	\node[draw,circle,fill=r]   ()   at (0.4,-1.5) 		{};	
%	\node[draw,circle,fill=j]	()   at (0.25,-1.9) 	{};	
%	\node[draw,circle,fill=r]   ()   at (-0.4,-1.5) 	{};	
%    \node[draw,circle,fill=v]	()   at (-0.25,-1.9) 	{};	
%	% 1-6-2
%	\node[draw,circle,fill=j]   ()   at (-0.7,1) 		{};	
%	\node[draw,circle,fill=r]	()   at (-1.2,1.0) 		{};	
%	\node[draw,circle,fill=j]   ()   at (-1.55,0.55) 	{};	
%	\node[draw,circle,fill=b]	()   at (-1.9,0.7) 		{};	
%	% 1-8-5
%	\node[draw,circle,fill=v]   ()   at (0.7,1) 		{};	
%	\node[draw,circle,fill=r]	()   at (1.2,1.0) 		{};	
%	\node[draw,circle,fill=v]   ()   at (1.55,0.55) 	{};	
%	\node[draw,circle,fill=b]	()   at (1.9,0.7) 		{};	
%	% 3-6 / 4-8
%	\node[draw,circle,fill=v]   ()   at (-1.55,0) 		{};	
%	\node[draw,circle,fill=b]	()   at (-1.2,-0.3) 	{};	
%	\node[draw,circle,fill=j]   ()   at (1.55,0) 		{};	
% 	\node[draw,circle,fill=b]	()   at (1.2,-0.3) 		{};	
    %complement
    \path (6) edge[looseness=1.5, bend left=90] (8);
    \path (6) edge[looseness=1, bend left=64] (8);
	\node[draw,circle,fill=b,scale=0.75]	(9)   at (90:2.4) 	{9};	
	\node[draw,circle,fill=v,scale=0.6]	(10)   at (110:2.4) {10};	
	\node[draw,circle,fill=b,scale=0.6]	(11)   at (90:3.4) 	{11};	
	\node[draw,circle,fill=j,scale=0.6]	(12)   at (70:3.3) 	{12};	
	\node[draw,circle,fill=r,scale=0.6]	(13)   at (90:2.9) 	{13};
	
	\path (1) edge[bicolor={b}{r},line width=4pt] (9);
	\path (9) edge[bicolor={b}{r},line width=4pt] (13);
	\path (13) edge[bicolor={b}{r},line width=4pt] (11);
	\path (7) edge[bicolor={b}{r},line width=4pt,looseness=1.7, bend left=90] (11);
	
	%\node[draw] at (\radium,\radium) {permutation 1-3};
\end{tikzpicture}	
			\caption{}\label{fig:cas4_etap1_3}
		\end{subfigure}
	\end{center}
	\caption{}
	\end{changemargin}
\end{figure}	
\FloatBarrier

Une fois cette permutation réalisée, il n'existe que 2 cas de figures : \\
\end{description}

%  ******************************************************
% Les 2 soit ......
%  ******************************************************
\begin{adjustwidth}{3.5em}{0pt}
\begin{description}
%  ******************************************************
%    soit 1/2
%  ******************************************************
\item[Soit : nous ne sommes plus dans le cas 4].\\
L’objectif est atteint !\\
Exemples de graphes :
\end{description}
\FloatBarrier
\begin{figure}[!ht]\centering
	\begin{changemargin}{-3cm}{-3cm}
		\begin{center}
			\begin{tikzpicture}
	%line width=5pt pour les ligne droite
	\def \n {5}
	\def \a {90}
	\def \x {360/\n * (\s - 1)+\a}
	\def \radius {1.1cm}
	\def \radium {2.2cm}
	\node[draw,circle,fill=w] (0) at (0,0) {0};
	% les 5 noeud interieur
	\def \s {1} \node[draw,circle,fill=r]	(\s)  at ({\x}:\radius)	{\s};	\path (\s) edge (0);
	\def \s {2} \node[draw,circle,fill=b]	(\s)  at ({\x}:\radius)	{\s};	\path (\s) edge (0);	\path (\s) edge (1);
	\def \s {3} \node[draw,circle,fill=v]	(\s)  at ({\x}:\radius)	{\s};	\path (\s) edge (0);	\path (\s) edge (2);
	\def \s {4} \node[draw,circle,fill=j]	(\s)  at ({\x}:\radius)	{\s};	\path (\s) edge (0);	\path (\s) edge[line width=3pt] (3);
	\def \s {5} \node[draw,circle,fill=b]	(\s)  at ({\x}:\radius)	{\s};	\path (\s) edge (0);	\path (\s) edge (4);	\path (\s) edge (1);
	% les 3 arc exterieur
	\draw[line width=3pt] (0,0) ++(168:\radium) arc (168:18:\radium);
	\draw (0,0) ++(18:\radium) arc (18:-90:\radium);
	%arc en 2 morceau...
	\draw (0,0) ++(-90:\radium) arc (-90:-180:\radium);
	\draw (0,0) ++(180:\radium) arc (180:168:\radium);
	% les 3 noeds exterieur	
	\def \t {1} \node[draw,circle,fill=j]	(6)   at (162:\radium) {6};		
	\def \t {2} \node[draw,circle,fill=r]	(7)   at (270:\radium) 	{7};		
	\def \t {3} \node[draw,circle,fill=v]	(8)   at (18:\radium) 	{8};		
	\path (1) edge (6);				
	\path (1) edge (8);			
	\path (2) edge (6);		
	\path (3) edge[line width=3pt] (6);
	\path (3) edge (7);			
	\path (4) edge (7); 			
	\path (4) edge[line width=3pt] (8);		
	\path (5) edge (8);		
	%\node[draw] at (\radium,\radium) {permutation 1-3};
\end{tikzpicture}
			\hspace{15pt}
			\input{./graphe_r/cas4_r1_1}
		\end{center}
		\end{changemargin}
	\begin{changemargin}{3cm}{0cm}
	\caption{la permutation de la chaine 1-2 sort du cas 4, le sommet 1 n'a plus de chaine commune avec les sommets 3 et 4}\label{fig:cas4_etape1_4}
\end{changemargin}	
\end{figure}	
\FloatBarrier
\begin{figure}[!ht]\centering
	\begin{changemargin}{0cm}{-3cm}
		\begin{center}
			\input{./graphe_r/cas4_r20_4}
			\hspace{15pt}
			\input{./graphe_r/cas4_r20_5}
		\end{center}
	\end{changemargin}
	\begin{changemargin}{3cm}{0cm}
	\caption{la permutation de la chaine 1-2 sort du cas 4, le sommet 1 n'a plus de chaine commune avec le sommets 4}\label{fig:cas4_etape1_5}
	\end{changemargin}
\end{figure}	
\FloatBarrier
Nous pouvons maintenant limiter le nombre de couleurs autour des voisins du sommet 0 à 3 par une des méthodes de \textsc{Kempe} (par exemple la $\mathcal{PC}_{13}$ ou la $\mathcal{PC}_{14}$).\\
\end{adjustwidth}

%  ******************************************************
% soit 2/2
%  ******************************************************
\begin{adjustwidth}{3.5em}{0pt}
\begin{description}
	\item[Soit : il existe au moins un sommet que nous appelons x, ayant les conditions suivantes :]\
	\begin{itemize}[label=--]
		\item il est à l'intersection de la $\overline{\mathcal{C}}_{13}$ et la $\overline{\mathcal{C}}_{14}$.
		\item la $\overline{\mathcal{C}}_{x3}$ n'a pas d'autre intersection avec la  $\overline{\mathcal{C}}_{14}$ que le sommet «~x~».\\
	\end{itemize}

\textit{L'ensemble de la démonstration ci-dessous se fait en analysant le comportement de chaines (et de permutations de chaines) comprenant le sommet x, 2 et 3. Il pourrait tout aussi bien être fait en symétrie sur le même sommet «~x~», mais avec les sommets~4 et 5}.
\\
\\
La première condition est simple, elle découle de notre première permutation. S'il n'existe plus d'intersection entre la  $\overline{\mathcal{C}}_{13}$ et la  $\overline{\mathcal{C}}_{14}$, nous ne sommes plus dans le cas 4.\\
La 2\up{eme} condition découle également de notre 1\up{ere} permutation. Si un sommet «~x~» a une chaine commune avec le sommet 3, et que cette chaine a un élément commun avec la  $\overline{\mathcal{C}}_{14}$ (de la couleur du sommet x) que l'on nomme «~y~», alors :
\begin{itemize}[label=--]
	\item
soit la  $\overline{\mathcal{C}}_{y2}$ est vide (figure~\ref{fig:cas4_etape1_6} gauche), et la  $\mathcal{PC}_{12}$ (1er étape) casse la  $\overline{\mathcal{C}}_{14}$ et nous ne sommes plus dans le cas 4 (figure~\ref{fig:cas4_etape1_6} droite). 
\begin{figure}[!ht]\centering
	\begin{changemargin}{0cm}{-3cm}
		\begin{center}
			\input{./graphe_r/cas4_r20_6}
			\hspace{15pt}
			\input{./graphe_r/cas4_r20_7}
		\end{center}
	\end{changemargin}
	\begin{changemargin}{3cm}{0cm}
	\caption{}\label{fig:cas4_etape1_6}
	\end{changemargin}
\end{figure}	
\FloatBarrier
	\item
soit la $\overline{\mathcal{C}}_{y2}$ existe (figure~\ref{fig:cas4_etape1_7} gauche) et «~y~» a les propriétés de «~x~» ! après la $\mathcal{PC}_{12}$ (figure~\ref{fig:cas4_etape1_7} droite) on est toujours dans le cas 4 et la $\overline{\mathcal{C}}_{y3}$ n'est pas vide et n'a plus d'intersection avec la $\overline{\mathcal{C}}_{14}$ (autre que «~y~»).
\begin{figure}[!ht]\centering
	\begin{changemargin}{0cm}{-3cm}
		\begin{center}
			\input{./graphe_r/cas4_r20_8}
			\hspace{15pt}
			\begin{tikzpicture}
	\def \n {5}
	\def \a {90}
	\def \x {360/\n * (\s - 1)+\a}
	\def \radius {1.1cm}
	\def \radium {2.6cm}
	\def \axesize {3.0cm}
	\draw[{\axeline}] (0,{0-\axesize}) -- (0,{\axesize});
	\node[draw,circle,fill=w] (0) at (0,0) {0};
	% les 5 noeud interieur
	\def \s {1} \node[draw,circle,fill=b]	(\s)  at ({\x}:\radius)	{\s};	\path (\s) edge (0);
	\def \s {2} \node[draw,circle,fill=r]	(\s)  at ({\x}:\radius)	{\s};	\path (\s) edge (0);	\path (\s) edge[bicolor={b}{r},line width=4pt] (1);
	\def \s {3} \node[draw,circle,fill=v]	(\s)  at ({\x}:\radius)	{\s};	\path (\s) edge (0);	\path (\s) edge (2);
	\def \s {4} \node[draw,circle,fill=j]	(\s)  at ({\x}:\radius)	{\s};	\path (\s) edge (0);	\path (\s) edge (3);
	\def \s {5} \node[draw,circle,fill=r]	(\s)  at ({\x}:\radius)	{\s};	\path (\s) edge (0);	\path (\s) edge (4);	
	\path (\s) edge[bicolor={b}{r},line width=4pt] (1);
	% les 3 arc exterieur
	\draw (0,0) ++(168:\radium) arc (168:18:\radium);
	\draw (0,0) ++(18:\radium) arc (18:-90:\radium);
	%arc en 2 morceau...
	\draw (0,0) ++(-90:\radium) arc (-90:-180:\radium);
	\draw (0,0) ++(180:\radium) arc (180:168:\radium);
	% les 3 noeds exterieur	
	\node[draw,circle,fill=j]	(6)   at (160:\radium) 	{6};		
	\node[draw,circle,fill=b]	(7)   at (-90:\radium)   {x};		
	\node[draw,circle,fill=v]	(8)   at (20:\radium) 	{8};		
	\node[draw,circle,fill=j,scale=0.5]	(12)   at (-120:\radium) 	{12};	
	\node[draw,circle,fill=b,scale=0.8,line width=2pt]	(14)   at (-150:\radium) 	{y};	
	\path (1) edge (6);				
	\path (1) edge (8);
	\path (2) edge (6);			
	\path (3) edge (6);
	%\path (3) edge[bicolor={v}{b},line width=4pt] (7); 			
	\path (4) edge (7);	
	%\path (4) edge (8);
	\path (4) edge[looseness=1.5,bend right=40] (8);	
	\path (5) edge (8);		
	\path (3) edge[bicolor={v}{r},line width=4pt] (14);	
	    %complement
	% 4-8
	\node[draw,circle,fill=r,scale=0.6]  (9)   at (1.45,-0.8)	{9};	
 	\node[draw,circle,fill=j,scale=0.5]	(10)   at (2.3,-0.2) 	{10};	
 	\node[draw,circle,fill=b,scale=0.5]	(11)   at (1.7,-0.1) 	{11};
 	
 	\node[draw,circle,fill=v,scale=0.5]	(13)   at (-1.8,-2.6) 	{13};
 	\node[draw,circle,fill=r,scale=0.5]	(15)   at (-1.0,-1.6) 	{15};
 	\path (7) edge[bicolor={b}{r},line width=4pt](15);	
 	\path (14) edge[bicolor={b}{r},line width=4pt](15);
 	\path (13) edge (14); 	
 	\path (7) edge (13); 	
 	
	\path (7) edge[bicolor={b}{r},line width=4pt] (9);	
	\path (11) edge[bicolor={b}{r},line width=4pt] (9);
	\path (5) edge[bicolor={b}{r},line width=4pt] (11);

\end{tikzpicture}	
		\end{center}
	\end{changemargin}
	\begin{changemargin}{3cm}{0cm}
	\caption{}\label{fig:cas4_etape1_7}
	\end{changemargin}
\end{figure}	
\FloatBarrier

\end{itemize}
\end{description}
\end{adjustwidth}

%  ******************************************************
% 2em etapes
%  ******************************************************
\begin{description}
\item[La 2\up{eme} étape] consiste à permuter la chaine commune 2-3  ($\mathcal{PC}_{23}$); ce qui nous amène à la configuration suivante :\	
\end{description}

\begin{figure}[!ht]\centering
	\begin{changemargin}{0cm}{-3cm}
		\begin{center}
			\input{./graphe_r/cas4_r6_2}
			\hspace{15pt}
			\input{./graphe_r/cas4_r10_1}
		\end{center}
	\end{changemargin}
	\begin{changemargin}{3cm}{0cm}
	\caption{}\label{fig:cas4_etape2_1}
	\end{changemargin}
\end{figure}	
\FloatBarrier


C’est le sommet 4 qui a maintenant 2 voisins de couleur identique. C’est ce même sommet 4 qui doit avoir 2 chaines communes avec les sommets 1 et 2 pour satisfaire la 1\up{ere}  condition du cas 4.
Or cette permutation crée une$\overline{\mathcal{C}}_{15}$ qui sépare et isole le sommet 4 du sommet 2  (figure~\ref{fig:cas4_etape2_1}).\\
Dans l'exemple des figures~\ref{fig:cas4_etape2_1}, nous sommes bien sorti du cas 4, la $\overline{\mathcal{C}}_{42}$ n'existe pas.\\
Si nous regardons plus en détail la $\overline{\mathcal{C}}_{x3}$, elle peut comme dans les 2 exemples ci-dessus ne comporter que 2 sommets (x et 3), mais elle peut également être constituée de multiples sommets.\\
Dans ce cas, est-il toujours possible de constituer une barrière « étanche » au passage de la $\overline{\mathcal{C}}_{42}$ qui nous permettra de sortir du cas 4 ?\\
Revenons à notre schéma après la 1\up{ere} permutation ($\mathcal{PC}_{12}$).
La $\overline{\mathcal{C}}_{x3}$ est de couleur «~x~» et 3. Par définition, on sait que la $\overline{\mathcal{C}}_{x3}$ n'a pas d’élément commun avec la $\overline{\mathcal{C}}_{14}$ (autre que x). La $\overline{\mathcal{C}}_{x3}$ est donc confinée à l’intérieur de la $\overline{\mathcal{C}}_{14}$ (figure~\ref{fig:cas4_etape2_barrage}).
\begin{figure}[!ht]\centering
	\begin{changemargin}{0cm}{-3cm}
		\begin{center}
			\input{./graphe_r/cas4_r40_3}
		\end{center}
	\end{changemargin}
	\begin{changemargin}{3cm}{0cm}
	\caption{}\label{fig:cas4_etape2_barrage}
	\end{changemargin}
\end{figure}	
\FloatBarrier


Pour autant qu'après la  $\mathcal{CP}_{23}$ il n'existe pas de $\overline{\mathcal{C}}_{x3}$, nous allons crée cette chaine par permutation successive.\\  

Nous appelons «~y~» un sommet sur cette chaîne dont la couleur est celle du sommet 3.
Nous parcourons tous les sommets «~y~» de la  $\overline{\mathcal{C}}_{x3}$ :
%  ******************************************************
% 2em etapes 1er soit
%  ******************************************************
\begin{adjustwidth}{3.5em}{0pt}
	\begin{description}
		\item[Soit la $\overline{\mathcal{C}}_{y3}$ est vide :]\
		
Dans ce cas (figure \ref{fig:cas4_etape2_2} gauche),  on peut réaliser une $\mathcal{PC}_{y3}$ sans que cela n'affecte ni le sommet 2 ni le sommet 5 (figure \ref{fig:cas4_etape2_2} droite) et donc aucun sommet voisin du sommet 0.
Le sommet 2 par définition (si cette chaine existe nous sommes dans les conditions du point suivant), le sommet 5 parce qu’il existe une $\overline{\mathcal{C}}_{14}$ (des 2 autres couleurs du graphe) qui sépare ces 2 sommets (figure~\ref{fig:cas4_etape2_barrage}).
\begin{figure}[!ht]\centering
	\begin{changemargin}{0cm}{-3cm}
		\begin{center}
			\input{./graphe_r/cas4_r40_6}
			\hspace{15pt}
			\input{./graphe_r/cas4_r40_7}
		\end{center}
	\end{changemargin}
	\begin{changemargin}{3cm}{0cm}
	\caption{}\label{fig:cas4_etape2_2}
	\end{changemargin}
\end{figure}	
\FloatBarrier

Chaque sommet ne devra être permuté qu'une seule fois. Dans le cas ou certains sommets ont une chaine commune avec la couleur du sommet 2 (dans la figure \ref{fig:cas4_etape2_3}, les sommets 12 et 13), il seront permuter ensemble (figure \ref{fig:cas4_etape2_3} droite)
\begin{figure}[!ht]\centering
	\begin{changemargin}{0cm}{-3cm}
		\begin{center}
			\input{./graphe_r/cas4_r50_1}
			\hspace{15pt}
			\input{./graphe_r/cas4_r50_2}
		\end{center}
	\end{changemargin}
	\begin{changemargin}{3cm}{0cm}
	\caption{}\label{fig:cas4_etape2_3}
	\end{changemargin}
\end{figure}	
\FloatBarrier

Notons au passage que dans le cas d'un binôme comme dans l'exemple de la figure \ref{fig:cas4_etape2_bionome} (gauche), il existe déjà une chaine de la couleur de la $\overline{\mathcal{C}}_{x3}$ entre les éléments qui bordent ce binôme. La $\mathcal{PC}_{y2}$ modifie cette chaine mais la recrée sur l’élément en binôme (la$\overline{\mathcal{C}}_{x3}$ existe toujours après la $\mathcal{PC}_{y2}$ - figure \ref{fig:cas4_etape2_bionome} droite).
\begin{figure}[!ht]\centering
	\begin{changemargin}{0cm}{-3cm}
		\begin{center}
			\input{./graphe_r/cas4_r50_3}
			\hspace{15pt}
			\begin{tikzpicture}
	\def \n {5}
	\def \a {90}
	\def \x {360/\n * (\s - 1)+\a}
	\def \radius {1.1cm}
	\def \radium {2.6cm}
	\def \axesize {3.0cm}
	\draw[{\axeline}] (0,{0-\axesize}) -- (0,{\axesize});
	\node[draw,circle,fill=w] (0) at (0,0) {0};
	% les 5 noeud interieur
	\def \s {1} \node[draw,circle,fill=b]	(\s)  at ({\x}:\radius)	{\s};	\path (\s) edge (0);
	\def \s {2} \node[draw,circle,fill=r]	(\s)  at ({\x}:\radius)	{\s};	\path (\s) edge (0);	\path (\s) edge (1);
	\def \s {3} \node[draw,circle,fill=v]	(\s)  at ({\x}:\radius)	{\s};	\path (\s) edge (0);	\path (\s) edge (2);
	\def \s {4} \node[draw,circle,fill=j]	(\s)  at ({\x}:\radius)	{\s};	\path (\s) edge (0);	\path (\s) edge (3);
	\def \s {5} \node[draw,circle,fill=r]	(\s)  at ({\x}:\radius)	{\s};	\path (\s) edge (0);	\path (\s) edge (4);	
	\path (\s) edge (1);
	% les 3 arc exterieur
	\draw (0,0) ++(168:\radium) arc (168:18:\radium);
	\draw (0,0) ++(18:\radium) arc (18:-90:\radium);
	%arc en 2 morceau...
	\draw (0,0) ++(-90:\radium) arc (-90:-180:\radium);
	\draw (0,0) ++(180:\radium) arc (180:168:\radium);
	% les 3 noeds exterieur	
	\node[draw,circle,fill=j]	(6)   at (160:\radium) 	{6};		
	\node[draw,circle,fill=b]	(7)   at (-90:\radium)   {7};		
	\node[draw,circle,fill=v]	(8)   at (20:\radium) 	{8};		
	\path (1) edge (6);				
	\path (1) edge (8);
	\path (2) edge (6);			
	%\path (3) edge (6);
	\path (3) edge[looseness=1.5,bend right=-40] (6);	
	%\path (3) edge (7); 			
	\path (4) edge[looseness=1.5,bend right=-30] (7);	
	%\path (4) edge (8);
	\path (4) edge[looseness=1.5,bend right=40] (8);	
	\path (5) edge (8);		
    %complement
	% 4-8
	%\path (13) edge[looseness=1.5,bend right=-30] (15);
	\node[draw,circle,fill=r,scale=0.6]  (9)   at (1.45,-0.8)	{9};	
 	\node[draw,circle,fill=j,scale=0.5]	(10)   at (2.3,-0.2) 	{10};	
 	\node[draw,circle,fill=b,scale=0.5]	(11)   at (1.7,-0.1) 	{11};
	\path (7) edge[looseness=1.5,bend right=30] (9);	
	\path (9) edge (11);
	\path (5) edge (11);

 	\node[draw,circle,fill=b,scale=0.6]	(12)   at (-0.4,-1.5) 	{12};
 	\node[draw,circle,fill=r,scale=0.6]	(13)   at ( 0.2,-1.7) 	{13};	
 	\node[draw,circle,fill=v,scale=0.6]	(14)   at (-0.7,-2.0) 	{14};	

 	\path (3) edge (12); 		
 	\path (12) edge (13); 		
 	\path (12) edge (14); 		
 	\path (7) edge (13); 		
 	\path (7) edge (14);
 	\path (13) edge (14);
 	
 	
	\end{tikzpicture}	
		\end{center}
	\end{changemargin}
	\begin{changemargin}{3cm}{0cm}
	\caption{}\label{fig:cas4_etape2_bionome}
	\end{changemargin}
\end{figure}	
\FloatBarrier

%  ******************************************************
% 2em etapes 2em soit
%  ******************************************************
		\item[Soit il existe une $\overline{\mathcal{C}}_{y2}$.]\
Dans ce cas (figure \ref{fig:cas4_etape2_rien} a gauche), lors de la $\mathcal{PC}_{23}$ (figure \ref{fig:cas4_etape2_rien} a droite) que nous ferons à l’étape suivante , le sommet «~y~» va changer de couleur (puisqu’il fait partie de la chaine). A ce stade, et dans ces conditions il ne faut rien faire !
\begin{figure}[!ht]\centering
	\begin{changemargin}{0cm}{-3cm}
		\begin{center}
			\input{./graphe_r/cas4_r30_1}
			\hspace{15pt}
			\input{./graphe_r/cas4_r30_2}
		\end{center}
	\end{changemargin}
	\begin{changemargin}{3cm}{0cm}
	\caption{}\label{fig:cas4_etape2_rien}
	\end{changemargin}
\end{figure}	
\FloatBarrier

	\end{description}
\end{adjustwidth}

%  ******************************************************
% 3em etapes
%  ******************************************************
\begin{description}
\item[La 3\up{eme} étape] consiste à permuter la chaine 2-3($\mathcal{PC}_{23}$) (figure~\ref{fig:cas4_etape3}).
Cette permutation (après les traitements préalables) a pour effet de créer une $\overline{\mathcal{C}}_{13}$ qui sépare le sommet 4 du sommet 2.
Dès lors, on est sorti du cas 4 puisqu'il n'y a plus de $\overline{\mathcal{C}}_{42}$ (impossible à réaliser de par la présence de la $\overline{\mathcal{C}}_{13}$).
\begin{figure}[!ht]\centering
	\begin{changemargin}{-3cm}{-3cm}
		\begin{center}
			\input{./graphe_r/cas4_r40_8}
		\end{center}
		\end{changemargin}
	\begin{changemargin}{3cm}{0cm}
	\caption{$\mathcal{PC}_{23}$ de la figure~\ref{fig:cas4_etape2_1} de droite}\label{fig:cas4_etape3}
\end{changemargin}	
\end{figure}	
\FloatBarrier


\end{description}	
