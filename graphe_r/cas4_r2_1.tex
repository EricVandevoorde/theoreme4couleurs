\begin{tikzpicture}
	%line width=5pt pour les ligne droite
	\def \n {5}
	\def \a {90}
	\def \x {360/\n * (\s - 1)+\a}
	\def \radius {1.1cm}
	\def \radium {2.2cm}
	\node[draw,circle,fill=w] (0) at (0,0) {0};
	% les 5 noeud interieur
	\def \s {1} \node[draw,circle,fill=r]	(\s)  at ({\x}:\radius)	{\s};	\path (\s) edge (0);
	\def \s {2} \node[draw,circle,fill=b]	(\s)  at ({\x}:\radius)	{\s};	\path (\s) edge (0);	\path (\s) edge (1);
	\def \s {3} \node[draw,circle,fill=v]	(\s)  at ({\x}:\radius)	{\s};	\path (\s) edge (0);	\path (\s) edge (2);
	\def \s {4} \node[draw,circle,fill=j]	(\s)  at ({\x}:\radius)	{\s};	\path (\s) edge (0);	\path (\s) edge[line width=3pt] (3);
	\def \s {5} \node[draw,circle,fill=b]	(\s)  at ({\x}:\radius)	{\s};	\path (\s) edge (0);	\path (\s) edge (4);	\path (\s) edge (1);
	% les 3 arc exterieur
	\draw[line width=3pt] (0,0) ++(168:\radium) arc (168:18:\radium);
	\draw (0,0) ++(18:\radium) arc (18:-90:\radium);
	%arc en 2 morceau...
	\draw (0,0) ++(-90:\radium) arc (-90:-180:\radium);
	\draw (0,0) ++(180:\radium) arc (180:168:\radium);
	% les 3 noeds exterieur	
	\def \t {1} \node[draw,circle,fill=j]	(6)   at (162:\radium) {6};		
	\def \t {2} \node[draw,circle,fill=r]	(7)   at (270:\radium) 	{7};		
	\def \t {3} \node[draw,circle,fill=v]	(8)   at (18:\radium) 	{8};		
	\path (1) edge (6);				
	\path (1) edge (8);			
	\path (2) edge (6);		
	\path (3) edge[line width=3pt] (6);
	\path (3) edge (7);			
	\path (4) edge (7); 			
	\path (4) edge[line width=3pt] (8);		
	\path (5) edge (8);		
	%\node[draw] at (\radium,\radium) {permutation 1-3};
\end{tikzpicture}