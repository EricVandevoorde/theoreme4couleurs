\documentclass[french]{report}
\usepackage[utf8]{inputenc}
\usepackage[T1]{fontenc}
\usepackage{lmodern}
\usepackage[a4paper]{geometry}
\usepackage[francais]{babel}
\usepackage{amsmath}
\usepackage{gensymb}
\usepackage{amsfonts}
\usepackage{amssymb}
\usepackage{sidecap}
\usepackage{xfrac}
\usepackage{caption}
\usepackage{hyperref}
\usepackage{afterpage}
\usepackage{soul}
\usepackage[bottom]{footmisc}
\usepackage{tikz}
\usepackage{boxedminipage}
\usepackage{chngcntr}
\usepackage{comment}
\usepackage{placeins} %\FloatBarrier
\counterwithout{equation}{chapter}
\counterwithout{figure}{chapter}
\usepackage{titlesec}
\usepackage{changepage}% http://ctan.org/pkg/changepage
 \usepackage{enumitem}

\usetikzlibrary{decorations}
\usepackage{subcaption}
%%%%%%%%%%%%%%%%%%%%%%%%%%%%%%%%%%%
\tikzset{
bicolor/.style 2 args={
  thick,dashed,dash pattern=on 6pt off 6pt,-,#1,
  postaction={draw,dashed,dash pattern=on 4pt off 8pt,-,#2,dash phase=5pt}
  },
}
%%%%%%%%%%%%%%%%%%%%%%%%%%%%%%%%%%%

%%%%%%%%%%%%%%%%%%%%%%%%%%%%%%%%%%%
%\titlespacing{\chapter} {0pt} {*0} {*0} {} 
%\titlespacing{\section} {0ex} {*0} {*0} {} 
%\titlespacing{\subsection} {0ex} {*0} {*0} {} 
%\titlespacing{\subsubsection} {10ex} {8ex} {*0} {} 

\begin{document}

	\def \axeline {dotted}
	\def\changemargin#1#2{\list{}{\rightmargin#2\leftmargin#1}\item[]}
	%\newcommand*{\Line}[3][]{\tikz \draw[#1] #2 -- #3;}
	\let\endchangemargin=\endlist 

\definecolor{b}{RGB}{100,150,200}
\definecolor{v}{RGB}{120,200,80}
\definecolor{r}{RGB}{220,80,70}
\definecolor{j}{RGB}{255,220,50}
\definecolor{f}{RGB}{180,80,180}
\definecolor{w}{RGB}{255,255,255}
\definecolor{n}{RGB}{0,0,0}

\begin{titlepage}
\begin{center}
\textit{}\vfill

\textit{}\\\textit{}\\\textit{}\\\textit{}\\
\today\\[0.8cm]
%\LARGE{\textsc{Université Libre de Bruxelles}}\\
%\LARGE{\textsc{Université Libre de Bruxelles}}\\
%\LARGE{Faculté des Sciences} \large{· Département de Physique}\\[0.8cm]

\rule{\linewidth}{.5pt}\\[1.0cm]

%{ \large {Mémoire présenté en vue de l'obtention\\du diplôme de Master en Sciences Physiques}}\\[0.8cm]
%{ \Huge \bfseries Magnétosphères\\autour de trous-noirs}\\[0.8cm]
{ \Huge \bfseries Théorème des Quatre Couleurs}\\[0.8cm]
%\Large{Travail dirigé par M. Geoffrey \textsc{Compère}}\\
%{\large Physique Théorique Mathématique}\\[0.2cm]

\rule{\linewidth}{.5pt} \\[1.2cm]

\LARGE{Éric \& Noé \textsc{Vandevoorde}}\\
%\large{avec l'aide de Noé \textsc{Vandevoorde}}
%\large{Année académique 2015·2016}

\vfill\textit{}
\end{center}
\end{titlepage}

\newpage
\pagenumbering{gobble}

\tableofcontents

 
% *********************************************************
% *********************************************************
% *********************************************************
\input{./theoremeIntroduction.tex}
\input{./theoreme5couleurs.tex}
\input{./theoreme4couleurs.tex}

% *********************************************************
% *********************************************************	
% *********************************************************
\section{Conclusion}
Tous les cas de figure ont été analysés et dans tous les cas, on peut montrer qu'il est possible de sortir du cas 4 par quelques permutations de chaines. Le fait de sortir du cas 4 nous ramène a une structure du graphe qui peut être traitée par une des 3 méthodes de \textsc{Kemps}. 
La démonstration est dès lors complète !

\end{document}